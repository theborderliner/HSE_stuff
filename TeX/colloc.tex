\documentclass[a4paper,11pt]{report}

\usepackage[T2A]{fontenc}
\usepackage[utf8]{inputenc}
\usepackage[english,russian]{babel}
\usepackage{amsmath} %математические пакеты
\usepackage{amsfonts}
\usepackage{amssymb}
\usepackage[pdftex,unicode]{hyperref}
\usepackage[top=2cm,
left=2cm,
right=2cm,
bottom=2cm]{geometry}
%\renewcommand{\familydefault}{\sfdefault}

\title{Коллок по линалу}
\author{Пешехонов Иван. БПМИ1912}
\date{\today}

\pdfinfo{%
  /Title    (Test document)
  /Author   (Ivan Peshekhonov)
  /Creator  (Ivan Peshekhonov)
  /Producer (Ivan Peshekhonov)
  /Subject  (Test)
  /Keywords (test)
}

\DeclareMathOperator{\real}{\rm I\!R}
\DeclareMathOperator{\Mnm}{\real^{n\times m}}
\DeclareMathOperator{\Mmn}{\real^{m\times n}}
\DeclareMathOperator{\Mn}{{\rm I\!R}^{n}}

\begin{document}

\maketitle %создает титульный лист
\tableofcontents
\clearpage
\chapter{Определения}
\section{Сумма двух матриц, произведение матрицы на скаляр}
\textbf{Сложение.} $A, B \in \Mnm, A = (a_{ij}), B = (b_{ij}) $
\[
A + B = (a_{ij} + b_{ij}) = 
\begin{pmatrix}
a_{11} + b_{11} & a_{12} + b_{12} & a_{13} + b_{13} & \cdots & a_{1n} + b_{1n}\\
a_{21} + b_{21} & a_{22} + b_{22} & a_{23} + b_{23} & \cdots & a_{2n} + b_{2n}\\
\vdots & \vdots & \vdots & \vdots \\
a_{m1} + b_{m1} & a_{m2} + b_{m2} & a_{m3} + b_{m3} & \cdots & a_{mn} + b_{mn}\\
\end{pmatrix}
\in \Mnm
\]
\textbf{Умножение на скаляр.} $A \in \Mnm, \lambda \in \real, A = (a_{ij}) \Rightarrow$
\[
\lambda{A} = (\lambda{a_{ij}}) =
\begin{pmatrix}
\lambda{a_{11}} & \lambda{a_{12}} & \cdots & \lambda{a_{1n}}\\
\lambda{a_{21}} & \lambda{a_{22}} & \cdots & \lambda{a_{2n}}\\
\vdots & \vdots & \vdots & \vdots \\
\lambda{a_{m1}} & \lambda{a_{m2}} & \cdots & \lambda{a_{mn}}\\
\end{pmatrix}
\in \Mnm
\]
\section{Транспонированная матрица}
Пусть $A \in \Mnm, A = (a_{ij})$
\[
A =
\begin{pmatrix}
a_{11} & a_{12} & \cdots & a_{1n}\\
a_{21} & a_{22} & \cdots & a_{2n}\\
\vdots & \vdots & \vdots & \vdots\\
a_{m1} & a_{m2} & \cdots & a_{mn}\\
\end{pmatrix}
\]
тогда транспонированная к A матрица (обозначается) $A^T$:
\[
A^T = 
\begin{pmatrix}
a_{11} & a_{21} & \cdots & a_{m1}\\
a_{12} & a_{22} & \cdots & a_{m2}\\
\vdots & \vdots & \vdots & \vdots\\
a_{1n} & a_{2n} & \cdots & a_{mn}\\
\end{pmatrix}
\]
\section{Произведение двух матриц}
$A \in \Mnm $,
$B \in \real^{m\times p}$\\
Тогда AB \textendash такая матрица $C \in \real^{n\times p}$, что $c_{ij} = A_{(i)}B^{(j)} = \sum_{k=1}^{n} a_{ik}b_{kj}$
\section{Диагональная матрица, умножение на диагональную матрицу слева и справа}
Матрица $A \in \Mnm$ называется \textbf{диагональной} $\Leftrightarrow$
\[
A = 
\begin{pmatrix}
a_1 & 0 & 0 & \cdots & 0 & 0\\
0 & a_2 & 0 & \cdots & 0 & 0\\
\vdots & \vdots & \ddots & \vdots & \vdots & \vdots\\
0 & 0 & 0 & a_n & \cdots & 0\\
\end{pmatrix}
= diag(a_1, a_2,\cdots, a_n)
\]

То есть
\[
\forall i, j \in \mathbb{N} \Rightarrow a_{ij} = 
\begin{cases}
  a_i & i = j\\
  0 & i \neq j
\end{cases}
\] 
\newline
\newline
Пусть A = $diag(a_1, a_2, \cdots, a_n) \in \Mn $, тогда \\

$
(1) B \in \Mnm \Rightarrow AB = 
\begin{pmatrix}
a_1{B_{(1)}}\\
a_2{B_{(2)}}\\
\vdots\\
a_n{B_{(n)}}\\
\end{pmatrix}
$ (Каждая строка $B$ умножается на соответсвующий элемент столбца матрицы $A$)

$
(2) B \in \Mnm \Rightarrow BA = 
\begin{pmatrix}
a_1{B^{(1)}} & a_2{B^{(2)}} & \cdots & a_n{B^{(n)}}\\
\end{pmatrix}
$ (Каждый сролбец $B$ умножается на соответсвующий элемент строки матрицы $A$)\\
\section{Единичная матрица, её свойства}
Матрица $A \in \Mn$ называется \textbf{единичной} $\Leftrightarrow$ $A = diag(1, 1, \cdots, 1) = 
\begin{pmatrix}
1 & 0 & \cdots & 0\\
0 & 1 & \cdots & 0\\
\vdots & \vdots & \ddots & \vdots\\
0 & 0 & \cdots & 1\\
\end{pmatrix}
$, обозначается $E$ (или $I$).\\
\\
\textbf{Свойства:}\\
(1) $EA = AE = A, \forall A \in \Mn$\\
(2) $E = E^{-1}$\\
\section{След квадратной матрицы и его поведение при сложении матриц, умножении матрицы на скаляр и транспонировании}
\textbf{Следом матрицы} называется сумма элементов её главной диагонали и обозначается $tr(A)$.\\
\\
\textbf{Свойства:}\\
(1) $tr(A + B) = tr(A) + tr(B)$\\
(2) $tr(\lambda{A}) = \lambda * tr(A)$\\
(3) $tr(A) = tr(A^T)$\\
\section{След произведения двух матриц}
$tr(AB) = tr(BA) \forall A \in \Mnm, B \in \Mmn$\\
\textbf{Доказательство.}\\
Пусть $X = AB, Y = BA$, тогда\\
$tr(X) = \sum_{i=1}^{n} x_{ii} = \sum_{i=1}^{n}\sum_{j=1}^{m} a_{ij}b_{ji} = 
\sum_{j=1}^{m}\sum_{i=1}^{n} b_{ji}a_{ij} =\sum_{j=1}^{m} y_{jj} = tr(Y) $ $\blacksquare$\\
\section{Совместные и несовместные системы линейных уравнений}
Система линейных уравнений (СЛУ):\\
\[
\begin{cases}
  a_{11}x_1 + a_{12}x_2 + \cdots + a_{1n}x_n = b_1\\
  a_{21}x_1 + a_{22}x_2 + \cdots + a_{2n}x_n = b_2\\
  \cdots\cdots\cdots\cdots\cdots\cdots\cdots\cdots\cdots\cdots\cdots\\
  a_{m1}x_1 + a_{m2}x_2 + \cdots + a_{mn}x_n = b_m\\
\end{cases}
\]
\textbf{Решением СЛУ} является такой набор значений неизвестных, который является решением каждого уравнения в СЛУ.\\
\\
CЛУ называется \textbf{совместной} если она имеет хотя бы одно решение. 
В противном случае СЛУ называется \textbf{несовместной}.\\
\section{Эквивалентные системы линейных уравнений}
Две СЛУ от \underline{одних и тех же переменных} называются \textbf{эквивалентыми} если у них совпадают множества решений.
\section{Расширенная матрица линейных уравнений}
\[
(*) = 
\begin{cases}
  a_{11}x_1 + a_{12}x_2 + \cdots + a_{1n}x_n = b_1\\
  a_{21}x_1 + a_{22}x_2 + \cdots + a_{2n}x_n = b_2\\
  \cdots\cdots\cdots\cdots\cdots\cdots\cdots\cdots\cdots\cdots\cdots\\
  a_{m1}x_1 + a_{m2}x_2 + \cdots + a_{mn}x_n = b_m\\
\end{cases}
\]
\textbf{Расширенной матрицей} СЛУ (\textasteriskcentered) называется матрица вида 
$\begin{pmatrix}A \textbar b\end{pmatrix} = 
\begin{pmatrix}
\begin{tabular}{1111|1}
a_{11} & a_{12} & \cdots & a_{1n} & b_1\\
a_{21} & a_{22} & \cdots & a_{2n} & b_2\\
\vdots & \vdots & \vdots & \vdots & \vdots\\
a_{m1} & a_{m2} & \cdots & a_{mn} & b_m\\
\end{tabular}
\end{pmatrix}$,
где $A$ \textendash матрица коэффициентов при неизвестных, 
а $b$ \textendash вектор-слобец правых частей каждого уравнения СЛУ (\textasteriskcentered).
\section{Элементарные преобразования строк матрицы}
\textbf{Элементарными преобразованиями} называют следующие три преобразрования, меняющие вид матрицы:\\
\[
\begin{tabular}{1|1|1}
1 тип & К i-ой строке матрицы прибавить j-ую, умноженную на \lambda & \CYREREV_1{(i, j, \lambda)}\\
2 тип & Поменять местами i-ую и j-ую строки местами & \CYREREV_2{(i, j)}\\
3 тип & i-ую строку матрицы умножить на \underline{ненулевую} \lambda & \CYREREV_3{(i, \lambda)}, \lambda \neq 0\\
\end{tabular}
\]
\section{Ступенчатый вид матрицы}

Строка $(a_1, a_2, \cdots, a_i)$ называется \textbf{нулевой}, если $a_1 = a_2 = \cdots = a_i = 0$, 
и \textbf{ненулевой} в обратном случае $(\exists i:  a_i \neq 0)$.\\
\\
\textbf{Ведущим элементом} называется первый ненулевой элемент нулевой строки.\\
\\
Матрица $A \in \Mnm$ называется \textbf{ступенчатой} или имеет \textbf{ступенчатый вид}, если:\\
1) Номера ведущих элементов строго возрастают.\\
2) Все нулевые строки расположены в конце.
\[
\begin{pmatrix}
 0 & \heartsuit & * & * & * & \cdots & * & *\\
 0 & 0 & 0 & \heartsuit & * & * & \cdots & *\\
 0 & 0 & 0 & 0 & \heartsuit & * & \cdots & *\\
 \vdots & \vdots & \vdots & \vdots & \vdots & \vdots & \vdots & \vdots\\
 0 & 0 & 0 & 0 & 0 &  \cdots & 0 & \heartsuit\\
 0 & 0 & 0 & 0 & 0 & 0 & \cdots & 0\\
 0 & 0 & 0 & 0 & 0 & 0 & \cdots & 0\\
\end{pmatrix}
\]
где $*$ \textendash что угодно, $\heartsuit \textendash \neq 0$
\section{Улучшеный ступенчатый вид матрицы}
Говорят, что матрица имеет \textbf{улучшенный (усиленный) ступенчатый вид}, если:\\
1) Она имеет ступенчатый вид.\\
2) Все ведущие элементы матрицы равны $1$.\\
\[
\begin{pmatrix}
 0 & 1 & * & 0 & 0 & \cdots & 0 & 0\\
 0 & 0 & 0 & 1 & 0 & 0 & \cdots & 0\\
 0 & 0 & 0 & 0 & 1 & * & \cdots & 0\\
 \vdots & \vdots & \vdots & \vdots & \vdots & \vdots & \vdots & \vdots\\
 0 & 0 & 0 & 0 & 0 &  \cdots & 0 & 1\\
 0 & 0 & 0 & 0 & 0 & 0 & \cdots & 0\\
 0 & 0 & 0 & 0 & 0 & 0 & \cdots & 0\\
\end{pmatrix}
\]
\section{Теорема о виде, к которому можно привести матрицу при помощи элементарных преобразований}
\textbf{Теорема 1.} Любую матрицу можно привести к ступенчатому виду.\\
\textbf{Доказательство:}\\

\end{document}