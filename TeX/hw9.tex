\documentclass[a4paper,11pt]{report}

\usepackage[T2A]{fontenc}
\usepackage[utf8]{inputenc}
\usepackage[english,russian]{babel}
\usepackage{amsmath} %математические пакеты
\usepackage{amsfonts}
\usepackage{amssymb}
\usepackage[pdftex,unicode]{hyperref}
\usepackage[dvipsnames]{xcolor}

\usepackage[top=2cm,
left=2cm,
right=2cm,
bottom=2cm]{geometry}
%\renewcommand{\familydefault}{\sfdefault}

\title{Коллок по линалу}
\author{Пешехонов Иван. БПМИ1912}
\date{\today}

\begin{document}
\chapter{Домашнее задание №9}
\section{№ 1.}
Для каждой клетки есть два исхода: мы её либо красим, либо нет, $\Leftrightarrow$ есть $2^n$ способо закрасить $n$ клеток.\\
$|$ Закрашено всё кроме верхнего ряда $|$ = $2^8$ $(*)$\\
$|$ Закрашено всё кроме нижнего ряда $|$ = $2^8$ (аналогично) $(\square)$\\
$|$ Закрашено всё кроме центральных столбцов $|$ = $2^6$ $(\Diamond)$\\
$* \bigcap \square = 2^4$ (только вторая строчка)\\
$* \bigcap \Diamond = 2^4$ (центральные ряды, нижние две строчки)\\
$\square \bigcap \Diamond = 2^4$ (центральные ряды, верхние две строчки)\\
$* \bigcap \Diamond \bigcap \square = 2^2$ (центральные столбцы, центральная строчка, т.е. две клетки)\\
Дальше по формуле включений и исключений:\\
$* \bigcup \Diamond \bigcup \square = 2^8 +2^8 + 2^6 - 3 * 2^4 + 2^2 
= 256 + 256 + 64 - 16 - 16 - 16 + 4 = 532$\\
\section{№ 2.}
Пусть $\mathbb{P}$ и $\mathbb{M}$ - множества пилотов и медиков. 
(Нам хватит только двух, потому что все профессии равнозначны)\\
Тогда посчитаем количество людей, которые умеют и кухарить и лечить\\
$|\mathbb{P} \bigcup \mathbb{M}| = |\mathbb{P}| + |\mathbb{M}| - |\mathbb{P} \bigcap \mathbb{M}| = 6 + 6 - 4 = 8$\\
Теперь посчитаем количество многозадачных людей:\\
$|$Человеки-оркестры$| = 4|\mathbb{P}| - 6|\mathbb{P} \bigcap \mathbb{M}| + 
4|\mathbb{P} \bigcap \mathbb{M} \bigcap \mathbb{F}_1| - 
|\mathbb{P} \bigcap \mathbb{M} \bigcap \mathbb{F}_1 \bigcap \mathbb{F}_2| =
6 * 4 - 6 * 4 + 4*2 - 1 = 7$\\
И короче прикол в чём, у нас есть 8 человек, которые умеют готовить и лечить, 
при этом вроде как достаточко 7 многозадачников, а значит ТЗ невыполнимо.
Но заказчику обычно пофиг.
\section{№ 3.}
Подумаем сначала, что может быть в $B$. Допустим, в $B$ больше элементов, чем в $A$,
тогда нельзя поставить в соответствие каждому элементу $B$ хотя бы один элемент из $A$,
т.е. не получится построить сюръекции. Если же в $B$ элементов меньше, чем в $A$, тогда
ровно те же проблемы возникают с построением инъекции. Ну а если элементов поровну, то 
мы очевидно говорим о числей способов построить биективное отображение, и в этому случае 
таких способов будет $n!$, потому что первому элементу множеству $A$ мы можем поставить 
в соответствие $n$ элементов из $B$, второму элементу уже $n - 1$, следующему $n - 2$ и так 
далее до $n$-ого, которому мы можем поставить в соответствие только один элемент из $B$, 
поскольку остальные заняты.\\
Круто конечно, но есть ещё один случай: когда $|A| = n$, а $|B| = 0$, тогда мы не можем 
построить ни инъекции, ни сюръекции, т.е. их одинаковое число.\\
\section{№ 4.}
Блин, а как рисовать графы в техе? А, не важно.\\
Вобщем имеем двуцветный граф на 20 вершинах. Цветом \textcolor{ForestGreen}{F} будем
красить рёбра между людьми, которые дружат, а цветом \textcolor{red}{E} будем 
красить рёбра между людьми, которые не дружат. Заметим, что по условию
каждая из двадцати вершин имеет 6 друзей и (20 - 6 - 1) не друзей.
Т.е. из каждой вершины выходят 6 рёбер цвета \textcolor{ForestGreen}{F}, и 13 ребёр
цвета \textcolor{red}{E}. Всего компаний по три человека у нас будет $
\begin{pmatrix}
20\\
3
\end{pmatrix}
= \frac{20!}{3! * 17!} = \frac{20 * 19 * 18}{3!} = 1140
$.\\
На графе такая группа представлена в виде треугольников, в вершинах которого... ну... вершины...
Причём треугольник
может иметь как один цвет, так и два. И кстатии, количество двуцвтных треугольников
мы можем посчитать следующим образом:\\
$\frac{20 * 13 * 6}{2!} = \frac{1560}{2} = 780$\\
Соотвественно, зная количество двуцветных треугольников (количество групп, в которых
не все дружат), мы можем однозначтно вычислить количество одноцветных треугольнико 
(количество групп, в которых все со всеми дружат):\\
$1140 - 780 = 360$.\\
Ответ: 360; 780.
\section{№ 8.}
а) Есть $n$ \underline{разных} конфет и $m$ \underline{разных} коробок.\\
Тогда\\
б) 
\end{document}
