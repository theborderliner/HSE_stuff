\documentclass[a4paper,11pt]{report}

\usepackage[T2A]{fontenc}
\usepackage[utf8]{inputenc}
\usepackage[english,russian]{babel}
\usepackage{amsmath} %математические пакеты
\usepackage{amsfonts}
\usepackage{amssymb}
\usepackage[pdftex,unicode]{hyperref}
\usepackage[dvipsnames]{xcolor}

\usepackage[top=2cm,
left=2cm,
right=2cm,
bottom=2cm]{geometry}

\title{Индивидуальное домашнее задание № 3.}
\author{Иван Пешехонов. БПМИ1912.}

\begin{document}
\maketitle
\chapter{Вариант 21.}
\section{№ 1.}
$
A = 
\begin{pmatrix}
-1 & 1 & -2 & 4 & 6\\
2 & -5 & 4 & -11 & -12\\
-2 & -1 & -3 & 3 & 2\\
2 & 2 & 3 & -2 & -2\\
-3 & 1 & -5 & 8 & 8\\
\end{pmatrix}
$\\

Первое что стоит сделать, это найти $rkA$, этим и займёмся:\\
\\
$
\begin{pmatrix}
-1 & 1 & -2 & 4 & 6\\
2 & -5 & 4 & -11 & -12\\
-2 & -1 & -3 & 3 & 2\\
2 & 2 & 3 & -2 & -2\\
-3 & 1 & -5 & 8 & 8\\
\end{pmatrix}
\rightsquigarrow
\begin{pmatrix}
-1 & 1 & -2 & 4 & 6\\
0 & -3 & 0 & -3 & 0\\
0 & -3 & 1 & -5 & -10\\
0 & 4 & -1 & 6 & 10\\
0 & -2 & 1 & -4 & -10\
\end{pmatrix}
\rightsquigarrow
\begin{pmatrix}
-1 & 1 & -2 & 4 & 6\\
0 & -3 & 0 & -3 & 0\\
0 & 0 & 1 & -2 & -10\\
0 & 0 & 1 & -2 & -10\\
0 & -2 & 1 & -4 & -10\
\end{pmatrix}
\rightsquigarrow
\\\\\\
\rightsquigarrow
\begin{pmatrix}
-1 & 1 & -2 & 4 & 6\\
0 & 1 & 0 & 1 & 0\\
0 & 0 & 1 & -2 & -10\\
0 & 0 & 0 & 0 & 0\\
0 & -2 & 1 & -4 & -10\
\end{pmatrix}
\rightsquigarrow
\begin{pmatrix}
-1 & 1 & -2 & 4 & 6\\
0 & 1 & 0 & 1 & 0\\
0 & 0 & 1 & -2 & -10\\
0 & 0 & 0 & 0 & 0\\
0 & 0 & 1 & -2 & -10\
\end{pmatrix}
\rightsquigarrow
\begin{pmatrix}
\underline{-1} & 1 & -2 & 4 & 6\\
0 & \underline{1} & 0 & 1 & 0\\
0 & 0 & \underline{1} & -2 & -10\\
0 & 0 & 0 & 0 & 0\\
0 & 0 & 0 &0 & 0\
\end{pmatrix}
$\\
(Впринцыпе, любые элементарные преобразования не меняют ранга матрицы, но мешать виды элементарных
преобразований в этой задаче это хороший способ выстрелить себе в ногу, поэтому я использовал исключительно
элементарные преобразования строк.)\\
Итак, $rkA = 3$. Значит, в результате нужно получить сумму \underline{трёх} матриц ранга $1$.\\
Полученную матрицу $A'$ можно представить как сумму трёх матриц ранга 1:\\
\\
$
A' = 
\begin{pmatrix}
\textcolor{red}{-1} & \textcolor{red}{1} & \textcolor{red}{-2} & \textcolor{red}{4} & \textcolor{red}{6}\\
\textcolor{Green}{0} & \textcolor{Green}{1} & \textcolor{Green}{0} & \textcolor{Green}{1} & \textcolor{Green}{0}\\
\textcolor{blue}{0} & \textcolor{blue}{0} & \textcolor{blue}{1} & \textcolor{blue}{-2} & \textcolor{blue}{-10}\\
0 & 0 & 0 & 0 & 0\\
0 & 0 & 0 &0 & 0\\
\end{pmatrix}
= 
\begin{pmatrix}
\textcolor{red}{-1} & \textcolor{red}{1} & \textcolor{red}{-2} & \textcolor{red}{4} & \textcolor{red}{6}\\
0 & 0 & 0 & 0 & 0\\
0 & 0 & 0 & 0 & 0\\
0 & 0 & 0 & 0 & 0\\
0 & 0 & 0 & 0 & 0\\
\end{pmatrix}
+
\begin{pmatrix}
0 & 0 & 0 & 0 & 0\\
\textcolor{Green}{0} & \textcolor{Green}{1} & \textcolor{Green}{0} & \textcolor{Green}{1} & \textcolor{Green}{0}\\
0 & 0 & 0 & 0 & 0\\
0 & 0 & 0 & 0 & 0\\
0 & 0 & 0 & 0 & 0\\
\end{pmatrix}
+
\begin{pmatrix}
0 & 0 & 0 & 0 & 0\\
0 & 0 & 0 & 0 & 0\\
\textcolor{blue}{0} & \textcolor{blue}{0} & \textcolor{blue}{1} & \textcolor{blue}{-2} & \textcolor{blue}{-10}\\
0 & 0 & 0 & 0 & 0\\
0 & 0 & 0 & 0 & 0\\
\end{pmatrix}
$\\
\\
Матрица $A'$ получена из матрицы $A$ путём цепочки элементарных преобразований строк $\Rightarrow$ матрица $A$ получается
из матрицы $A'$ путём цепочки обратных элементарных преобразований.\\
Т.к. матрица $A'$ представима в виде суммы трёх матриц ранга 1, то если применить к каждой из таких матриц цепочку обратных элементарных преобразований, получится искомоеразложение матрицы $A$.\\
Введём обозначение:\\
$A' = \textcolor{red}{A^{\sharp}} + \textcolor{Green}{A^{\flat}} + \textcolor{blue}{A^{\natural}}$\\
Теперь выполняем для каждой матрицы ранга 1 цепочку обратных элементарных преобразований:\\
$A^{\sharp} = 
\begin{pmatrix}
-1 & 1 & -2 & 4 & 6\\
0 & 0 & 0 & 0 & 0\\
0 & 0 & 0 & 0 & 0\\
0 & 0 & 0 & 0 & 0\\
0 & 0 & 0 & 0 & 0\\
\end{pmatrix}
$\\
\section{№ 3.}
$U$ - подпространство в $\mathbb{R}^4$, такое что 
\[
\langle
\begin{pmatrix}
4\\
1\\
-9\\
-17\\
\end{pmatrix},
\begin{pmatrix}
3\\
-1\\
-5\\
-4\\
\end{pmatrix},
\begin{pmatrix}
7\\
1\\
-15\\
-26\\
\end{pmatrix},
\begin{pmatrix}
-1\\
-3\\
5\\
18\\
\end{pmatrix}
\rangle
= U
\]
Составим ОСЛУ из этих векторов:\\
\[
\begin{cases}
4x_1 + x_2 - 9x_3 - 17x_4 = 0\\
3x_1 - x_2 - 5x_3 - 4x_4 = 0\\
7x_1 + x_2 - 15x_3 - 26x_4 = 0\\
-x_1 - 3x_2 + 5x_3 + 18x_4 = 0\\
\end{cases}
\Leftrightarrow
\begin{pmatrix}
4 & 1 & -9 & -17\\
3 & -1 & -5 & -4\\
7 & 1 & -15 & -26\\
-1 & -3 & 5 & 18\\
\end{pmatrix}
\begin{pmatrix}
x_1\\
x_2\\
x_3\\
x_4\\
\end{pmatrix}
=
0
\]
Найдём ФСР:\\
\\
$
\begin{pmatrix}
\begin{tabular}{c c c c | c}
4 & 1 & -9 & -17 & 0\\
3 & -1 & -5 & -4 & 0\\
7 & 1 & -15 & -26 & 0\\
-1 & -3 & 5 & 18 & 0\\
\end{tabular}
\end{pmatrix}
\rightsquigarrow
\begin{pmatrix}
\begin{tabular}{c c c c | c}
0 & -11 & 11 & 55 & 0\\
0 & -10 & 10 & 50 & 0\\
0 & -20 & 20 & 100 & 0\\
-1 & -3 & 5 & 18 & 0\\
\end{tabular}
\end{pmatrix}
\rightsquigarrow
\begin{pmatrix}
\begin{tabular}{c c c c | c}
0 & -1 & 1 & 5 & 0\\
0 & -1 & 1 & 5 & 0\\
0 & -1 & 1 & 5 & 0\\
-1 & -3 & 5 & 18 & 0\\
\end{tabular}
\end{pmatrix}
\rightsquigarrow
\\\\\\
\rightsquigarrow
\begin{pmatrix}
\begin{tabular}{c c c c | c}
-1 & -3 & 5 & 18 & 0\\
0 & -1 & 1 & 5 & 0\\
0 & 0 & 0 & 0 & 0\\
0 & 0 & 0 & 0 & 0\\
\end{tabular}
\end{pmatrix}
\rightsquigarrow
\begin{pmatrix}
\begin{tabular}{c c c c | c}
1 & 3 & -5 & -18 & 0\\
0 & 1 & -1 & -5 & 0\\
0 & 0 & 0 & 0 & 0\\
0 & 0 & 0 & 0 & 0\\
\end{tabular}
\end{pmatrix}
\rightsquigarrow
\begin{pmatrix}
\begin{tabular}{c c c c | c}
1 & 0 & -2 & -3 & 0\\
0 & 1 & -1 & -5 & 0\\
0 & 0 & 0 & 0 & 0\\
0 & 0 & 0 & 0 & 0\\
\end{tabular}
\end{pmatrix}
\Rightarrow
\\\\\\
\Rightarrow
\left\{
\begin{pmatrix}
2x_3 + 3x_4\\
x_3 + 5x_4\\
x_3\\
x_4\\
\end{pmatrix}
|\ 
x_3, x_4 \in \mathbb{R}
\right\}
$
\\\\\\
ФСР:	
$
\begin{pmatrix}
2\\
1\\
1\\
0\\
\end{pmatrix},
\begin{pmatrix}
3\\
5\\
0\\
1\\
\end{pmatrix}
$
\\\\
Из ФСР по строкам составим ОСЛУ:\\
\[
\begin{cases}
2x_1 + x_2 + x_3 = 0\\
3x_1 + 5x_2 + x_4 = 0\\
\end{cases}
\]
\\
Всё, эта ОСЛУ - искомая.
\section{№ 4.}
$L_1 = 
\langle
\begin{pmatrix}
4\\
-1\\
-14\\
9\\
\end{pmatrix},
\begin{pmatrix}
0\\
1\\
-4\\
3\\
\end{pmatrix},
\begin{pmatrix}
-6\\
-4\\
-11\\
6\\
\end{pmatrix},
\begin{pmatrix}
2\\
2\\
1\\
0\\
\end{pmatrix}
\rangle$
\\
\\
$L_2 = 
\langle
\begin{pmatrix}
-2\\
0\\
-9\\
6\\
\end{pmatrix},
\begin{pmatrix}
8\\
-3\\
21\\
-13\\
\end{pmatrix},
\begin{pmatrix}
4\\
-3\\
3\\
-1\\
\end{pmatrix},
\begin{pmatrix}
10\\
-6\\
15\\
-8\\
\end{pmatrix}
\rangle$\\\\
$U = L_1 + L_2 =
\langle
a_1, a_2, a_3, a_4, b_1, b_2, b_3, b_4
\rangle$\\
Найдём базисы $L_1, L_2, U$:\\
Запишем векторы в столбцы и будем выполнять элементарные преобразования строк, сохраняя линейные зависимости между столбцами.
Запишем матрицу $(a_1\ a_2\ a_3\ a_4\ |\ b_1\ b_2\ b_3\ b_4)$. Таким образом мы стразу найдём базис $L_1$ и $U$:\\
\\
$
\begin{pmatrix}
\begin{tabular}{c c c c | c c c c}
4 & 0 & -6 & 2 & -2 & 8 & 4 & 10\\
-1 & 1 & -4 & 2 & 0 & -3 & -3 & -6\\
-14 & -4 & -11 & 1 & -9 & 21 & 3 & 15\\
9 & 3 & 6 & 0 & 6 & -13 & -1 & -8\\
\end{tabular}
\end{pmatrix}
\rightsquigarrow
\begin{pmatrix}
\begin{tabular}{c c c c | c c c c}
0 & 4 & -22 & 10 & -2 & -4 & -8 & -14\\
-1 & 1 & -4 & 2 & 0 & -3 & -3 & -6\\
0 & -18 & 45 & -27 & -9 & 63 & 45 & 99\\
0 & 12 & -30 & 18 & 6 & -40 & -28 & -62\\
\end{tabular}
\end{pmatrix}
\rightsquigarrow
\\\\
\rightsquigarrow
\begin{pmatrix}
\begin{tabular}{c c c c | c c c c}
0 & 2 & -11 & 5 & -1 & -2 & -4 & -7\\
-1 & 1 & -4 & 2 & 0 & -3 & -3 & -6\\
0 & -6 & 15 & -9 & -3 & 21 & 15 & 33\\
0 & 6 & -15 & 9 & 3 & -20 & -14 & -31\\
\end{tabular}
\end{pmatrix}
\rightsquigarrow
\begin{pmatrix}
\begin{tabular}{c c c c | c c c c}
0 & 2 & -11 & 5 & -1 & -2 & -4 & -7\\
-1 & 1 & -4 & 2 & 0 & -3 & -3 & -6\\
0 & 2 & -5 & 3 & 1 & -7 & -5 & -11\\
0 & 0 & 0 & 0 & 0 & 1 & 1 & 2\\
\end{tabular}
\end{pmatrix}
\rightsquigarrow
\\\\
\rightsquigarrow
\begin{pmatrix}
\begin{tabular}{c c c c | c c c c}
0 & 0 & \underline{-6} & 2 & -2 & 5 & 1 & 4\\
\underline{-1} & 1 & -4 & 2 & 0 & -3 & -3 & -6\\
0 & \underline{2} & -5 & 3 & 1 & -7 & -5 & -11\\
0 & 0 & 0 & 0 & 0 & \underline{1} & 1 & 2\\
\end{tabular}
\end{pmatrix}
$\\
\\
Соответственно базис $L_1$: 
$
\begin{pmatrix}
4\\
-1\\
-14\\
9\\
\end{pmatrix},
\begin{pmatrix}
0\\
1\\
-4\\
3\\
\end{pmatrix},
\begin{pmatrix}
-6\\
-4\\
-11\\
6\\
\end{pmatrix}
$;\\
базис $U$:
$
\begin{pmatrix}
4\\
-1\\
-14\\
9\\
\end{pmatrix},
\begin{pmatrix}
0\\
1\\
-4\\
3\\
\end{pmatrix},
\begin{pmatrix}
-6\\
-4\\
-11\\
6\\
\end{pmatrix}
\begin{pmatrix}
8\\
-3\\
21\\
-13\\
\end{pmatrix}
$;\\
Найдём базис $L_2$: запишем векторы в матрицу по столбцам и будем выполнять элементарные преобразования строк, 
сохраняя линейные зависимости между столбцами:\\
\\
$
\begin{pmatrix}
-2 & 8 & 4 & 10\\
0 & -3 & -3 & -6\\
-9 & 21 & 3 & 15\\
6 & -13 & -1 & -8\\
\end{pmatrix}
\rightsquigarrow
\begin{pmatrix}
1 & -4 & -2 & -5\\
0 & 1 & 1 & 2\\
3 & -7 & -1 & -5\\
6 & -13 & -1 & -8\\
\end{pmatrix}
\rightsquigarrow
\begin{pmatrix}
1 & 0 & 2 & 3\\
0 & 1 & 1 & 2\\
3 & -7 & -1 & -5\\
0 & 1 & 1 & 2\\
\end{pmatrix}
\rightsquigarrow
\begin{pmatrix}
1 & 0 & 2 & 3\\
0 & 1 & 1 & 2\\
0 & -7 & -7 & -14\\
0 & 0 & 0 & 0\\
\end{pmatrix}
\rightsquigarrow
\\
\rightsquigarrow
\begin{pmatrix}
\underline{1} & 0 & 2 & 3\\
0 & \underline{1} & 1 & 2\\
0 & 0 & 0 & 0\\
0 & 0 & 0 & 0\\
\end{pmatrix}
$\\
\\
Базис в $L_2$: 
$
\begin{pmatrix}
-2\\
0\\
-9\\
6\\
\end{pmatrix},
\begin{pmatrix}
8\\
-3\\
21\\
-13\\
\end{pmatrix}
$.\\
\\
Прежде чем искать базис $W = L_1 \cap L_2$ разберёмся с размерностями:\\
1) $\dim{L_1} = 3$.\\
2) $\dim{L_2} = 2$.\\
3) $\dim{U} = 4$.\\
4) $\dim{W} = \dim{L_1} + \dim{L_2} - \dim{U} = 3 + 2 - 4 = 1$.\\
\\
Теперь ищем базис $W = L_1 \cap L_2$.\\
Рассмотрим некоторый вектор $v \in W$. Для такого вектора одновременно выполняется $v \in L_1$ и $v \in L_2$.\\
$v \in L_1 \Leftrightarrow \lambda_1a_1 + \lambda_2a_2 + \lambda_3a_3 = v$ (векторы $a_1, a_2, a_3$ - базис в $L_1$);\\
С другой стороны $v \in L_2 \Leftrightarrow \mu_1b_1 + \mu_2b_2 = v$ (векторы $b_1, b_2$ - базис в $L_2$).\\
Значит $v \in W \Leftrightarrow \lambda_1a_1 + \lambda_2a_2 + \lambda_3a_3 = \mu_1b_1 + \mu_2b_2 \Leftrightarrow
(a_1\ a_2\ a_3)
\begin{pmatrix}
\lambda_1\\
\lambda_2\\
\lambda_3\\
\end{pmatrix}
= 
(b_1\ b_2)
\begin{pmatrix}
\mu_1\\
\mu_2\\
\end{pmatrix}
\Leftrightarrow
\\
\Leftrightarrow
(a_1\ a_2\ a_3\ |\ b_1\ b_2)$. Ищем ФСР этой ОСЛУ:\\
\\
$
\begin{pmatrix}
\begin{tabular}{c c c | c c}
-4 & 0 & -6 & -2 & 8\\
-1 & 1 & -4 & 0 & -3\\
-14 & -4 & -11 & -9 & 21\\
9 & 3 & 6 & 6 & -13\\
\end{tabular}
\end{pmatrix}
\rightsquigarrow
\begin{pmatrix}
\begin{tabular}{c c c | c c}
0 & 0 & -6 & -2 & 5\\
-1 & 1 & -4 & 0 & -3\\
0 & 2 & -5 & 1 & -7\\
0 & 0 & 0 & 0 & 1\\
\end{tabular}
\end{pmatrix}
\rightsquigarrow
\begin{pmatrix}
\begin{tabular}{c c c | c c}
-1 & 1 & -4 & 0 & -3\\
0 & 2 & -5 & 1 & -7\\
0 & 0 & -6 & -2 & 5\\
0 & 0 & 0 & 0 & 1\\
\end{tabular}
\end{pmatrix}
\rightsquigarrow
\\
\rightsquigarrow
\begin{pmatrix}
\begin{tabular}{c c c | c c}
-1 & 1 & -4 & 0 & 0\\
0 & 2 & -5 & 1 & 0\\
0 & 0 & -6 & -2 & 0\\
0 & 0 & 0 & 0 & 1\\
\end{tabular}
\end{pmatrix}
\rightsquigarrow
\begin{pmatrix}
\begin{tabular}{c c c | c c}
1 & -1 & 4 & 0 & 0\\
0 & 2 & -5 & 1 & 0\\
0 & 0 & 3 & 1 & 0\\
0 & 0 & 0 & 0 & 1\\
\end{tabular}
\end{pmatrix}
\rightsquigarrow
\begin{pmatrix}
\begin{tabular}{c c c | c c}
1 & -1 & 4 & 0 & 0\\
0 & 2 & -8 & 0 & 0\\
0 & 0 & 3 & 1 & 0\\
0 & 0 & 0 & 0 & 1\\
\end{tabular}
\end{pmatrix}
\rightsquigarrow
\\
\rightsquigarrow
\begin{pmatrix}
\begin{tabular}{c c c | c c}
1 & -1 & 4 & 0 & 0\\
0 & 1 & -4 & 0 & 0\\
0 & 0 & 3 & 1 & 0\\
0 & 0 & 0 & 0 & 1\\
\end{tabular}
\end{pmatrix}
\rightsquigarrow
\begin{pmatrix}
\begin{tabular}{c c c | c c}
1 & 0 & 0 & 0 & 0\\
0 & 1 & -4 & 0 & 0\\
0 & 0 & 1& 1/3 & 0\\
0 & 0 & 0 & 0 & 1\\
\end{tabular}
\end{pmatrix}
\rightsquigarrow
\begin{pmatrix}
\begin{tabular}{c c c | c c}
1 & 0 & 0 & 0 & 0\\
0 & 1 & 0 & 4/3 & 0\\
0 & 0 & 1& 1/3 & 0\\
0 & 0 & 0 & 0 & 1\\
\end{tabular}
\end{pmatrix}
\Rightarrow\\
\Rightarrow$
ФСР:
$
\begin{pmatrix}
0\\
4\\
1\\
3\\
0\\
\end{pmatrix}$\\\\
Мы получили порождающую базис $W$ ОСЛУ: $0a_1 + 4a_2 + a_3 = 3b_1 + 0b_2$. Теперь подставим векторы в ОСЛУ и 
получим искомый базис:\\
$
4\cdot
\begin{pmatrix}
0\\
1\\
-4\\
3\\
\end{pmatrix}
+
\begin{pmatrix}
-6\\
-4\\
-11\\
6\\
\end{pmatrix}
=
3\cdot
\begin{pmatrix}
-2\\
0\\
-9\\
6\\
\end{pmatrix}
\Leftrightarrow
\begin{pmatrix}
-6\\
0\\
-27\\
18\\
\end{pmatrix}
=
\begin{pmatrix}
-6\\
0\\
-27\\
18\\
\end{pmatrix}
\Rightarrow\\
\Rightarrow
\begin{pmatrix}
-6\\
0\\
-27\\
18\\
\end{pmatrix}$
- базис в $W$. (Базисный вектор кстатии один, значит размерность равна 1, и это соответствует
нашим ожиданиям.)
\end{document}

