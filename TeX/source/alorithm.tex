\documentclass[a4paper,11pt]{report}
\usepackage[T2A]{fontenc}
\usepackage[utf8]{inputenc}
\usepackage[english,russian]{babel}
\usepackage{amsmath} %математические пакеты
\usepackage{amsfonts}
\usepackage{amssymb}
\usepackage[pdftex,unicode]{hyperref}
\usepackage[top=2cm,
left=2cm,
right=2cm,
bottom=2cm]{geometry}
\usepackage[dvipsnames]{xcolor}

\title{}
\author{Пешехонов Иван. БПМИ1912}
\date{\today}

\begin{document}
\chapter{Чудо-алгоритм.}
\textbf{Дано}:\\
$U, W$ - подпространства в $V$.\\
$U = \langle a_1\ a_2\ a_3\ a_4 \rangle$\\
$W = \langle b_1\ b_2\ b_3\ b_4 \rangle$\\
\textbf{Задача}:\\
Найти базис пересечения $U$ и $W$.\\
\section{Теория}
Нам заданы линейные оболочки подпространств, по определению: линейная оболочка пространства - множество всевозможных линейных комбинаций векторов из пространства.\\
Говорят, что вектор $v$ принадлежит линейной оболочке $L$, если существует линейная комбинация векторов из оболочки, равная $v$.\\
Формально: $v \in L \Leftrightarrow v = \lambda_n{l_1} + ... + \lambda_n{l_n}$,\\ 
где $l_1, ..., l_n \in L;\ \lambda_1, ..., \lambda_n$ - скаляры.\\
\\
Пересечение подпространств $U$ и $W$ содержит все такие векторы, которые одновеменно принадлежат и $U$, и $W$.
Т.е. одновеменно принадлежат и линейной оболочке $U$ и линейной оболочке $W$.\\
\\
Перепишем последний абзац:\\
Если $v \in U \cap W$, то $
\begin{cases}
 v \in U\\
 v \in W
\end{cases}
\Leftrightarrow
\begin{cases}
 v = a_1u_1 + ... + a_nu_n & u_1...u_n \in U\\
 v = b_1w_1 + ... + b_kw_k & w_1...w_k \in W\\
\end{cases}
$, а $a_i$ и $b_i$ - скаляры.\\
\\
Теперь мы можем записать это таким образом:\\
$a_1u_1 + ... + a_nu_n = v = b_1w_1 + ... + b_kw_k$, и убрать $v$:\\
$a_1u_1 + ... + a_nu_n =  b_1w_1 + ... + b_kw_k$.\\
\\
Полученное тождество является однородной системой линейных уравнений, покажем это, если не очень очевидно:\\
$a_1u_1 + ... + a_nu_n =  b_1w_1 + ... + b_kw_k $.\\
\\
$(u_1\ ...\ u_n)
\begin{pmatrix}
 a_1\\
 \vdots\\
 a_n
\end{pmatrix}
=  
(w_1\ ...\ w_n)
\begin{pmatrix}
 b_1\\
 \vdots\\
 b_n
\end{pmatrix}
$\\\\\\
$(u_1\ ...\ u_n)
\begin{pmatrix}
 a_1\\
 \vdots\\
 a_n
\end{pmatrix}
-(w_1\ ...\ w_n)
\begin{pmatrix}
 b_1\\
 \vdots\\
 b_n
\end{pmatrix}
=  0
$\\
\\\\\\\\
Важно осознать следующий переход:\\
$(u_1\ ...\ u_n\ -w_1\ ...\ -w_n)
\begin{pmatrix}
 a_1\\
 \vdots\\
 a_n\\
 b_1\\
 \vdots\\
 b_n
\end{pmatrix}
=  0
$\\
Запишем расширенную матрицу этой ОСЛУ:\\
$(u_1\ ...\ u_n\ -w_1\ ...\ -w_n\ |\ 0) \Leftrightarrow (u_1\ ...\ u_n\ |\ w_1\ ...\ w_n)$.\\
\\
Дальше мы решаем эту ОСЛУ как обычно: приводим к улучшенному ступенчатому виду (заметим, что если до этого мы находили базис в 
$U + W$, то у нас уже есть эта матрица, приведённая к ступенчатому виду, получается что теперь мы просто мучаем её дальше, и выигрываем
на этом какое-то время [дохуя времени на самом деле, мы же не делаем переход к системам]), выписываем общее решение, находим ФСР.\\
Дальше лучше показать на практике.\\
\section{Пример}
\[L_1 = 
\langle
\begin{pmatrix}
4\\
-1\\
-14\\
9\\
\end{pmatrix},
\begin{pmatrix}
0\\
1\\
-4\\
3\\
\end{pmatrix},
\begin{pmatrix}
-6\\
-4\\
-11\\
6\\
\end{pmatrix},
\begin{pmatrix}
2\\
2\\
1\\
0\\
\end{pmatrix}
\rangle,
\]
\[
L_2 = 
\langle
\begin{pmatrix}
-2\\
0\\
-9\\
6\\
\end{pmatrix}
\begin{pmatrix}
8\\
-3\\
21\\
-13\\
\end{pmatrix},
\begin{pmatrix}
4\\
-3\\
3\\
-1\\
\end{pmatrix},
\begin{pmatrix}
10\\
-6\\
15\\
-8\\
\end{pmatrix}
\rangle\]\\
Ну и с места в карьер: запишем расширенную матрицу $(L_1\ |\ L_2)$\\
\\
$
\begin{pmatrix}
\begin{tabular}{c c c c | c c c c}
 4 & 0 & -6 & 2 & -2 & 8 & 4 & 10\\
 -1 & 1 & -4 & 2 & 0 & -3 & -3 & -6\\
 -14 & -4 & -11 & 1 & -9 & 21 & 3 & 15\\
 9 & 3 & 6 & 0 & 6 & -13 & -1 & -1\\
\end{tabular}
\end{pmatrix}
\rightsquigarrow
\begin{pmatrix}
\begin{tabular}{c c c c | c c c c}
 -1 & 1 & -4 & 2 & 0 & -3 & -3 & -6\\
 0 & 2 & -5 & 3 & 1 & -7 & -5 & -11\\
 0 & 0 & -6 & 2 & -2 & 5 & 1 & 4\\
 0 & 0 & 0 & 0 & 0 & 1 & 1 & 2\\
\end{tabular}
\end{pmatrix}
$\\\\
На этом этапе мы привели матрицу к ступенчатому виду, и можем увидеть, что первый, второй, третий и шестой векторы в исходной матрице
образуют базис в $L_1 + L_2$. \\
Теперь мы не бросаем эту матрицу, а продолжаем работать с ней, приводя её к улучшенному ступенчатому виду:\\
\\
$
\begin{pmatrix}
\begin{tabular}{c c c c | c c c c}
 -1 & 1 & -4 & 2 & 0 & -3 & -3 & -6\\
 0 & 2 & -5 & 3 & 1 & -7 & -5 & -11\\
 0 & 0 & -6 & 2 & -2 & 5 & 1 & 4\\
 0 & 0 & 0 & 0 & 0 & 1 & 1 & 2\\
\end{tabular}
\end{pmatrix}
\rightsquigarrow
\begin{pmatrix}
\begin{tabular}{c c c c | c c c c}
 -1 & 0 & -1.5 & 0.5 & -0.5 & 0 & -1 & -1.5\\
 0 & 1 & 0 & 2/3 & !0.5 & !0 & !1 & !1.5\\
 0 & 0 & 1 & -1/3 & 1/3 & 0 & 2/3 & 1\\
 0 & 0 & 0 & 0 & 0 & 1 & 1 & 2\\
\end{tabular}
\end{pmatrix}
$\\\\
\end{document}
