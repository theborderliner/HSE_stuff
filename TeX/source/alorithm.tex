\documentclass[a4paper,11pt]{report}
\usepackage[T2A]{fontenc}
\usepackage[utf8]{inputenc}
\usepackage[english,russian]{babel}
\usepackage{amsmath} %математические пакеты
\usepackage{amsfonts}
\usepackage{amssymb}
\usepackage[pdftex,unicode]{hyperref}
\usepackage[top=2cm,
left=2cm,
right=2cm,
bottom=2cm]{geometry}
\usepackage[dvipsnames]{xcolor}

\title{}
\author{Пешехонов Иван. БПМИ1912}
\date{\today}

\begin{document}
\chapter{Чудо-алгоритм.}
\textbf{Дано}:\\
$U, W$ - подпространства в $V$.\\
$U = \langle a_1\ a_2\ a_3\ a_4 \rangle$\\
$W = \langle b_1\ b_2\ b_3\ b_4 \rangle$\\
\textbf{Задача}:\\
Найти базис пересечения $U$ и $W$.\\
\section{Теория}
Нам заданы линейные оболочки подпространств, по определению: линейная оболочка пространства - множество всевозможных линейных комбинаций векторов из пространства.\\
Говорят, что вектор $v$ принадлежит линейной оболочке $L$, если существует линейная комбинация векторов из оболочки, равная $v$.\\
Формально: $v \in L \Leftrightarrow v = \lambda_n{l_1} + ... + \lambda_n{l_n}$,\\ 
где $l_1, ..., l_n \in L;\ \lambda_1, ..., \lambda_n$ - скаляры.\\
\\
Пересечение подпространств $U$ и $W$ содержит все такие векторы, которые одновеменно принадлежат и $U$, и $W$.
Т.е. одновеменно принадлежат и линейной оболочке $U$ и линейной оболочке $W$.\\
\\
Перепишем последний абзац:\\
Если $v \in U \cap W$, то $
\begin{cases}
 v \in U\\
 v \in W
\end{cases}
\Leftrightarrow
\begin{cases}
 v = a_1u_1 + ... + a_nu_n & u_1...u_n \in U\\
 v = b_1w_1 + ... + b_kw_k & w_1...w_k \in W\\
\end{cases}
$, а $a_i$ и $b_i$ - скаляры.\\
\\
Теперь мы можем записать это таким образом:\\
$a_1u_1 + ... + a_nu_n = v = b_1w_1 + ... + b_kw_k$, и убрать $v$:\\
$a_1u_1 + ... + a_nu_n =  b_1w_1 + ... + b_kw_k$.\\
\\
Полученное тождество является однородной системой линейных уравнений, покажем это, если не очень очевидно:\\
$a_1u_1 + ... + a_nu_n =  b_1w_1 + ... + b_kw_k $.\\
\\
$(u_1\ ...\ u_n)
\begin{pmatrix}
 a_1\\
 \vdots\\
 a_n
\end{pmatrix}
=  
(w_1\ ...\ w_n)
\begin{pmatrix}
 b_1\\
 \vdots\\
 b_n
\end{pmatrix}
$\\\\\\
$(u_1\ ...\ u_n)
\begin{pmatrix}
 a_1\\
 \vdots\\
 a_n
\end{pmatrix}
-(w_1\ ...\ w_n)
\begin{pmatrix}
 b_1\\
 \vdots\\
 b_n
\end{pmatrix}
=  0
$\\
\\\\\\\\
Важно осознать следующий переход:\\
$(u_1\ ...\ u_n\ -w_1\ ...\ -w_n)
\begin{pmatrix}
 a_1\\
 \vdots\\
 a_n\\
 b_1\\
 \vdots\\
 b_n
\end{pmatrix}
=  0
$\\
Запишем расширенную матрицу этой ОСЛУ:\\
$(u_1\ ...\ u_n\ -w_1\ ...\ -w_n\ |\ 0) \Leftrightarrow (u_1\ ...\ u_n\ |\ w_1\ ...\ w_n)$.\\
\\
Дальше мы решаем эту ОСЛУ как обычно: приводим к улучшенному ступенчатому виду (заметим, что если до этого мы находили базис в 
$U + W$, то у нас уже есть эта матрица, приведённая к ступенчатому виду, получается что теперь мы просто мучаем её дальше, и выигрываем
на этом какое-то время [дохуя времени на самом деле, мы же не делаем переход к системам]), выписываем общее решение, находим ФСР.\\
Дальше лучше показать на практике.\\
\section{Пример}
\[L_1 = 
\langle
\begin{pmatrix}
4\\
-1\\
-14\\
9\\
\end{pmatrix},
\begin{pmatrix}
0\\
1\\
-4\\
3\\
\end{pmatrix},
\begin{pmatrix}
-6\\
-4\\
-11\\
6\\
\end{pmatrix},
\begin{pmatrix}
2\\
2\\
1\\
0\\
\end{pmatrix}
\rangle,
\]
\[
L_2 = 
\langle
\begin{pmatrix}
-2\\
0\\
-9\\
6\\
\end{pmatrix}
\begin{pmatrix}
8\\
-3\\
21\\
-13\\
\end{pmatrix},
\begin{pmatrix}
4\\
-3\\
3\\
-1\\
\end{pmatrix},
\begin{pmatrix}
10\\
-6\\
15\\
-8\\
\end{pmatrix}
\rangle\]\\
Ну и с места в карьер: запишем расширенную матрицу $(L_1\ |\ L_2)$\\
\\
$
\begin{pmatrix}
\begin{tabular}{c c c c | c c c c}
 4 & 0 & -6 & 2 & -2 & 8 & 4 & 10\\
 -1 & 1 & -4 & 2 & 0 & -3 & -3 & -6\\
 -14 & -4 & -11 & 1 & -9 & 21 & 3 & 15\\
 9 & 3 & 6 & 0 & 6 & -13 & -1 & -1\\
\end{tabular}
\end{pmatrix}
\rightsquigarrow
\begin{pmatrix}
\begin{tabular}{c c c c | c c c c}
 -1 & 1 & -4 & 2 & 0 & -3 & -3 & -6\\
 0 & 2 & -5 & 3 & 1 & -7 & -5 & -11\\
 0 & 0 & -6 & 2 & -2 & 5 & 1 & 4\\
 0 & 0 & 0 & 0 & 0 & 1 & 1 & 2\\
\end{tabular}
\end{pmatrix}
$\\\\
На этом этапе мы привели матрицу к ступенчатому виду, и можем увидеть, что первый, второй, третий и шестой векторы в исходной матрице
образуют базис в $L_1 + L_2$. Из них, кстатии, первый, второй и третий образуют базис в $L_1$.\\
Теперь мы не бросаем эту матрицу, а продолжаем работать с ней, приводя её к улучшенному ступенчатому виду:\\
\\
$
\begin{pmatrix}
\begin{tabular}{c c c c | c c c c}
 -1 & 1 & -4 & 2 & 0 & -3 & -3 & -6\\
 0 & 2 & -5 & 3 & 1 & -7 & -5 & -11\\
 0 & 0 & -6 & 2 & -2 & 5 & 1 & 4\\
 0 & 0 & 0 & 0 & 0 & 1 & 1 & 2\\
\end{tabular}
\end{pmatrix}
\rightsquigarrow
\begin{pmatrix}
\begin{tabular}{c c c c | c c c c}
 -1 & 0 & 0 & 0 & 0 & 0 & 0 & 0\\
 0 & 1 & 0 & 2/3 & -4/3 & 0 & -8/3 & -4\\
 0 & 0 & 1 & -1/3 & 1/3 & 0 & 2/3 & 1\\
 0 & 0 & 0 & 0 & 0 & 1 & 1 & 2\\
\end{tabular}
\end{pmatrix}
$\\\\\\\\\\\\
Выпишем общее решение:\\\\
$
\begin{pmatrix}
 0\\
 \frac{-2}{3}x_4 - \frac{4}{3}x_5 - \frac{8}{3}x_7 - 4x_8\\
 \frac{1}{3}x_4 - \frac{1}{3}x_5 - \frac{2}{3}x_7 - x_8\\
 x_4\\
 x_5\\
 -x_7 - 2x_8\\
 x_7\\
 x_8\\
\end{pmatrix}
$\\
\\
Очень важно разобраться со знаками, обратим внимание, что разделитель $|$ в матрице фактически обознает знак равентсва в уравнениях.\\
Соответственно, когда мы выражаем главные переменные через свободные, мы обычно переносим главные неизвесные в правую часть уравнения,
а свободные переменные - в левую.\\
Таким образом в выражении главных неизвесных свободные, стоящие слева от знака разделения ($x_4$) мы выписываем в общее решение с 
противоположным знаком (потому что как бы перекидываем их через равно), а свободные неизвесные, стоящие справа от разделителя ($x_5,\ x_7,\ x_8$)
мы записываем в общее решение не меняя их знак (потому что они итак как-бы стоят справа от равно).\\
По той же логике обратим внимание на то, что главная переменная $x_6$ выражается через свободные $x_7$ и $x_8$, при этом они стоят
с одной стороны от разделителя. Значит, свободные переменные нужно как-бы перекинуть через равно, сменив им знаки.\\
\\
Теперь запишем ФСР нашей ОСЛУ: (заметим, что нет строго правила, типа "В свободные неизвесные надо подставлять строго единицу``. 
Наример в этом примере, если мы подставим в переменную $x_4$ единицу, то в ФСР получится много дробей, я этого не хочу, поэтому буду подставлять
в $x_4$ число $3$)\\\\
\[
\begin{pmatrix}
 \textcolor{red}{0}\\
 \textcolor{red}{-2}\\
 \textcolor{red}{1}\\
 \textcolor{red}{3}\\
 \textcolor{blue}{0}\\
 \textcolor{blue}{0}\\
 \textcolor{blue}{0}\\
 \textcolor{blue}{0}\\
\end{pmatrix},
\begin{pmatrix}
 \textcolor{red}{0}\\
 \textcolor{red}{4}\\
 \textcolor{red}{1}\\
 \textcolor{red}{0}\\
 \textcolor{blue}{-3}\\
 \textcolor{blue}{0}\\
 \textcolor{blue}{0}\\
 \textcolor{blue}{0}\\
\end{pmatrix},
\begin{pmatrix}
 \textcolor{red}{0}\\
 \textcolor{red}{-8}\\
 \textcolor{red}{-2}\\
 \textcolor{red}{0}\\
 \textcolor{blue}{0}\\
 \textcolor{blue}{3}\\
 \textcolor{blue}{-3}\\
 \textcolor{blue}{0}\\
\end{pmatrix},
\begin{pmatrix}
 \textcolor{red}{0}\\
 \textcolor{red}{-4}\\
 \textcolor{red}{-1}\\
 \textcolor{red}{0}\\
 \textcolor{blue}{0}\\
 \textcolor{blue}{2}\\
 \textcolor{blue}{0}\\
 \textcolor{blue}{-1}\\
\end{pmatrix}
\]\\ 
Обозначим за $a_1\ a_2\ a_3\ a_4$ исходные векторы из $L_1$, а за $b_1\ b_2\ b_3\ b_4$ исходные векторы из $L_2$. Запишем ОСЛУ в явном виде:
\[
\textcolor{red}{\lambda_1}a_1 + \textcolor{red}{\lambda_2}a_2 + \textcolor{red}{\lambda_3}a_3 + \textcolor{red}{\lambda_4}a_4 
= 
\textcolor{blue}{\lambda_5}b_1 + \textcolor{blue}{\lambda_6}b_2 + \textcolor{blue}{\lambda_7}b_3 + \textcolor{blue}{\lambda_8}b_4
\]
Теперь важная вещь: мы берём первый вектор из ФСР: 
$
\begin{pmatrix}
 \textcolor{red}{0}\\
 \textcolor{red}{-2}\\
 \textcolor{red}{1}\\
 \textcolor{red}{3}\\
 \textcolor{blue}{0}\\
 \textcolor{blue}{0}\\
 \textcolor{blue}{0}\\
 \textcolor{blue}{0}\\
\end{pmatrix}
$, и делаем одну из двух вещей:\\
1) Подставляем красные координаты на места красных лямбд\\
2) Подставляем синие координаты на места синих лямбд\\
\\
\\
Проделав одну из этих операций мы получим вектор, который принадлежит $L_1 \cap L_2$.
Впринцыпе, для самопроверки можно сначала подставить красные координаты ФСР в красные лямбды, а синие координаты ФСР
в синие лямбды, и тогда, если слева и справа получился одинаковый вектор, то всё хорошо, а если не получился, значит где-то лажа.\\
Теперь, проделав одну из этих операций для каждого вектора из ФСР мы получим 4 вектора, которые образуют линейную оболочку подпространства
$L_1 \cap L_2$. Давайте теперь проделаем это. В синей секции больше нулей, поэтому я буду подставлять синие координаты и синие лямбды, просто чтобы
меньше считать:\\
Для первого вектора ФСР получаем $
\begin{pmatrix}
 0\\
 0\\
 0\\
 0\\
\end{pmatrix}
$\\
Для второго вектора ФСР: 
$
\begin{pmatrix}
 -6\\
 0\\
 -27\\
 18\\
\end{pmatrix}
$\\
Для третьего: 
$
\begin{pmatrix}
 12\\
 0\\
 54\\
 -36\\
\end{pmatrix}
$\\
Для четвёртого: 
$
\begin{pmatrix}
 6\\
 0\\
 27\\
 -18\\
\end{pmatrix}
$\\
\\
Мы получили четыре вектора, линейная оболочка которых образует подпространство $L_1 \cap L_2$. На этом собственно объяснение алгоритма
можно считать законченым, т.к. чтобы получить финальный ответ, достаточно уже приненить стандартный алгоритм выделения базиса в линейной оболочке.\\
Для красоты всё же досчитаем до конца: несложно заметить, что первый, второй и третий векторы пропорциональны четвёртому, следовательно
базисом пересечения $L_1$ и $L_2$ является вектор 
\[\begin{pmatrix}
 6\\
 0\\
 27\\
 -18\\
\end{pmatrix}\]
\section{Summarize}
1) Записываем расширенную матрицу $(L_1\ |\ L_2)$ (по столбцам, разумеется)\\
2) Приводим расширенную матрицу к ступенчатому виду\\
3) Приведя её к ступенчатому виду мы выполнили два действия одновременно: \underline{номера исходных векторов} со ступеньками,
расположенными \underline{левее разделителя} дают базис в $L_1$, а, соответственно,\\ \underline{номера исходных векторов} со
ступеньками во \underline{всей матрице} дают базис в $L_1 + L_2$.\\
4) Приводим ступенчатую матрицу дальше к улучшенному ступенчатому виду\\
5) Записываем общее решение (\textbf{следим за знаками!})\\
6) Записываем ФСР\\
7) Подставляем соответствующие координаты из векторов ФСР на места соответствующих скаляров в ОСЛУ (\textbf{следим за количеством!})\\
8) Получаем векторы, которые образовывают линейную оболочку подпространства $L_1 \cap L_2$\\
9) Применяем стандартный алгоритм нахождения базиса в линейной оболочке и находим базис подпространства $L_1 \cap L_2$\\
10) Profit!
\end{document}
