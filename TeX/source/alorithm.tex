\documentclass[a4paper,11pt]{report}
\usepackage[T2A]{fontenc}
\usepackage[utf8]{inputenc}
\usepackage[english,russian]{babel}
\usepackage{amsmath} %математические пакеты
\usepackage{amsfonts}
\usepackage{amssymb}
\usepackage[pdftex,unicode]{hyperref}
\usepackage[top=2cm,
left=2cm,
right=2cm,
bottom=2cm]{geometry}
\usepackage[dvipsnames]{xcolor}

\title{}
\author{Пешехонов Иван. БПМИ1912}
\date{\today}

\begin{document}
\chapter{Чудо-алгоритм.}
\textbf{Дано}:\\
$U, W$ - подпространства в $V$.\\
$U = \langle a_1\ a_2\ a_3\ a_4 \rangle$\\
$W = \langle b_1\ b_2\ b_3\ b_4 \rangle$\\
\textbf{Задача}:\\
Найти базис пересечения $U$ и $W$.\\
\section{Теория}
Нам заданы линейные оболочки подпространств, по определению: линейная оболочка пространства - множество всевозможных линейных комбинаций векторов из пространства.\\
Говорят, что вектор $v$ принадлежит линейной оболочке $L$, если существует линейная комбинация векторов из оболочки, равная $v$.\\
Формально: $v \in L \Leftrightarrow v = \lambda_n{l_1} + ... + \lambda_n{l_n}$,\\ 
где $l_1, ..., l_n \in L;\ \lambda_1, ..., \lambda_n$ - скаляры.\\
\\
Пересечение подпространств $U$ и $W$ содержит все такие векторы, которые одновеменно принадлежат и $U$, и $W$.
Т.е. одновеменно принадлежат и линейной оболочке $U$ и линейной оболочке $W$.\\
\\
Перепишем последний абзац:\\
Если $v \in U \cap W$, то $
\begin{cases}
 v \in U\\
 v \in W
\end{cases}
\Leftrightarrow
\begin{cases}
 v = a_1u_1 + ... + a_nu_n & u_1...u_n \in U\\
 v = b_1w_1 + ... + b_kw_k & w_1...w_k \in W\\
\end{cases}
$, а $a_i$ и $b_i$ - скаляры.\\
\\
Теперь мы можем записать это таким образом:\\
$a_1u_1 + ... + a_nu_n = v = b_1w_1 + ... + b_kw_k$, и убрать $v$:\\
$a_1u_1 + ... + a_nu_n =  b_1w_1 + ... + b_kw_k$.\\
\\
Полученное тождоество является однородной системой линейных уравнений, покажем это, если не очень очевидно:\\
$a_1u_1 + ... + a_nu_n =  b_1w_1 + ... + b_kw_k $.\\
\\
$(u_1\ ...\ u_n)
\begin{pmatrix}
 a_1\\
 \vdots\\
 a_n
\end{pmatrix}
=  
(w_1\ ...\ w_n)
\begin{pmatrix}
 b_1\\
 \vdots\\
 b_n
\end{pmatrix}
$\\\\\\
$(u_1\ ...\ u_n)
\begin{pmatrix}
 a_1\\
 \vdots\\
 a_n
\end{pmatrix}
-(w_1\ ...\ w_n)
\begin{pmatrix}
 b_1\\
 \vdots\\
 b_n
\end{pmatrix}
=  0
$\\
\\\\\\\\
Важно осознать следующий переход:\\
$(u_1\ ...\ u_n\ -w_1\ ...\ -w_n)
\begin{pmatrix}
 a_1\\
 \vdots\\
 a_n\\
 b_1\\
 \vdots\\
 b_n
\end{pmatrix}
=  0
$\\
Запишем расширенную матрицу этой ОСЛУ:\\
$(u_1\ ...\ u_n\ -w_1\ ...\ -w_n\ |\ 0) \Leftrightarrow (u_1\ ...\ u_n\ |\ w_1\ ...\ w_n)$.\\
\\
Дальше мы решаем эту ОСЛУ как обычно: приводим к улучшенному ступенчатому виду (заметим, что если до этого мы находили базис в 
$U + W$, то у нас уже есть эта матрица, приведённая к ступенчатому виду, получается что теперь мы просто мучаем её дальше, и выигрываем
на этом какое-то время [дохуя времени на самом деле, мы же не делаем переход к системам]), выписываем общее решение, находим ФСР.\\
Дальше лучше показать на практике.\\
\end{document}
