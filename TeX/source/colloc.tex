\documentclass[a4paper,11pt]{report}
\usepackage[T2A]{fontenc}
\usepackage[utf8]{inputenc}
\usepackage[english,russian]{babel}
\usepackage{amsmath} %математические пакеты
\usepackage{amsfonts}
\usepackage{amssymb}
\usepackage[pdftex,unicode]{hyperref}
\usepackage[top=2cm,
left=2cm,
right=2cm,
bottom=2cm]{geometry}
\usepackage[dvipsnames]{xcolor}

\title{Коллок по линалу}
\author{Пешехонов Иван. БПМИ1912}
\date{\today}

\pdfinfo{%
  /Title    (Test document)
  /Author   (Ivan Peshekhonov)
  /Creator  (Ivan Peshekhonov)
  /Producer (Ivan Peshekhonov)
  /Subject  (Test)
  /Keywords (test)
}

\DeclareMathOperator{\real}{\rm I\!R}
\DeclareMathOperator{\Mnm}{\real^{n\times m}}
\DeclareMathOperator{\Mmn}{\real^{m\times n}}
\DeclareMathOperator{\Mn}{{\rm I\!R}^{n}}

\begin{document}

\maketitle %создает титульный лист
\tableofcontents
\clearpage
\chapter{Определения}
\section{Сумма двух матриц, произведение матрицы на скаляр}
\textbf{Сложение.} $A, B \in \Mnm, A = (a_{ij}), B = (b_{ij}) $
\[
A + B = (a_{ij} + b_{ij}) = 
\begin{pmatrix}
a_{11} + b_{11} & a_{12} + b_{12} & a_{13} + b_{13} & \cdots & a_{1n} + b_{1n}\\
a_{21} + b_{21} & a_{22} + b_{22} & a_{23} + b_{23} & \cdots & a_{2n} + b_{2n}\\
\vdots & \vdots & \vdots & \vdots \\
a_{m1} + b_{m1} & a_{m2} + b_{m2} & a_{m3} + b_{m3} & \cdots & a_{mn} + b_{mn}\\
\end{pmatrix}
\in \Mnm
\]
\textbf{Умножение на скаляр.} $A \in \Mnm, \lambda \in \real, A = (a_{ij}) \Rightarrow$
\[
\lambda{A} = (\lambda{a_{ij}}) =
\begin{pmatrix}
\lambda{a_{11}} & \lambda{a_{12}} & \cdots & \lambda{a_{1n}}\\
\lambda{a_{21}} & \lambda{a_{22}} & \cdots & \lambda{a_{2n}}\\
\vdots & \vdots & \vdots & \vdots \\
\lambda{a_{m1}} & \lambda{a_{m2}} & \cdots & \lambda{a_{mn}}\\
\end{pmatrix}
\in \Mnm
\]
\section{Транспонированная матрица}
Пусть $A \in \Mnm, A = (a_{ij})$
\[
A =
\begin{pmatrix}
a_{11} & a_{12} & \cdots & a_{1n}\\
a_{21} & a_{22} & \cdots & a_{2n}\\
\vdots & \vdots & \vdots & \vdots\\
a_{m1} & a_{m2} & \cdots & a_{mn}\\
\end{pmatrix}
\]
тогда транспонированная к A матрица (обозначается) $A^T$:
\[
A^T = 
\begin{pmatrix}
a_{11} & a_{21} & \cdots & a_{m1}\\
a_{12} & a_{22} & \cdots & a_{m2}\\
\vdots & \vdots & \vdots & \vdots\\
a_{1n} & a_{2n} & \cdots & a_{mn}\\
\end{pmatrix}
\]
\section{Произведение двух матриц}
$A \in \Mnm $,ы
$B \in \real^{m\times p}$\\
Тогда AB \textendash такая матрица $C \in \real^{n\times p}$, что $c_{ij} = A_{(i)}B^{(j)} = \sum_{k=1}^{n} a_{ik}b_{kj}$
\section{Диагональная матрица, умножение на диагональную матрицу слева и справа}
\textcolor{red}{Квадратная} матрица $A \in$ \textcolor{red}{$\Mn$} называется \textbf{диагональной} $\Leftrightarrow$
\[
A = 
\begin{pmatrix}
a_1 & 0 & 0 & \cdots & 0 & 0\\
0 & a_2 & 0 & \cdots & 0 & 0\\
\vdots & \vdots & \ddots & \vdots & \vdots & \vdots\\
0 & 0 & 0 & a_n & \cdots & 0\\
\end{pmatrix}
= diag(a_1, a_2,\cdots, a_n)
\]

То есть
\[
\forall i, j \in \mathbb{N} \Rightarrow a_{ij} = 
\begin{cases}
  a_i & i = j\\
  0 & i \neq j
\end{cases}
\] 
\newline
\newline
Пусть A = $diag(a_1, a_2, \cdots, a_n) \in \Mn $, тогда \\

$
(1) B \in \Mnm \Rightarrow AB = 
\begin{pmatrix}
a_1{B_{(1)}}\\
a_2{B_{(2)}}\\
\vdots\\
a_n{B_{(n)}}\\
\end{pmatrix}
$ (Каждая строка $B$ умножается на соответсвующий элемент столбца матрицы $A$)

$
(2) B \in \Mnm \Rightarrow BA = 
\begin{pmatrix}
a_1{B^{(1)}} & a_2{B^{(2)}} & \cdots & a_n{B^{(n)}}\\
\end{pmatrix}
$ (Каждый сролбец $B$ умножается на соответсвующий элемент строки матрицы $A$)\\
\section{Единичная матрица, её свойства}
Матрица $A \in \Mn$ называется \textbf{единичной} $\Leftrightarrow$ $A = diag(1, 1, \cdots, 1) = 
\begin{pmatrix}
1 & 0 & \cdots & 0\\
0 & 1 & \cdots & 0\\
\vdots & \vdots & \ddots & \vdots\\
0 & 0 & \cdots & 1\\
\end{pmatrix}
$, обозначается $E$ (или $I$).\\
\\
\textbf{Свойства:}\\
(1) $EA = AE = A, \forall A \in \Mn$\\
(2) $E = E^{-1}$\\
\section{След квадратной матрицы и его поведение при сложении матриц, умножении матрицы на скаляр и транспонировании}
\textbf{Следом матрицы} называется сумма элементов её главной диагонали и обозначается $tr(A)$.\\
\\
\textbf{Свойства:}\\
(1) $tr(A + B) = tr(A) + tr(B)$\\
(2) $tr(\lambda{A}) = \lambda * tr(A)$\\
(3) $tr(A) = tr(A^T)$\\
\section{След произведения двух матриц}
$tr(AB) = tr(BA) \forall A \in \Mnm, B \in \Mmn$\\
\textbf{Доказательство.}\\
Пусть $X = AB, Y = BA$, тогда\\
$tr(X) = \sum_{i=1}^{n} x_{ii} = \sum_{i=1}^{n}\sum_{j=1}^{m} a_{ij}b_{ji} = 
\sum_{j=1}^{m}\sum_{i=1}^{n} b_{ji}a_{ij} =\sum_{j=1}^{m} y_{jj} = tr(Y) $ $\blacksquare$\\
\section{Совместные и несовместные системы линейных уравнений}
Система линейных уравнений (СЛУ):\\
\[
\begin{cases}
  a_{11}x_1 + a_{12}x_2 + \cdots + a_{1n}x_n = b_1\\
  a_{21}x_1 + a_{22}x_2 + \cdots + a_{2n}x_n = b_2\\
  \cdots\cdots\cdots\cdots\cdots\cdots\cdots\cdots\cdots\cdots\cdots\\
  a_{m1}x_1 + a_{m2}x_2 + \cdots + a_{mn}x_n = b_m\\
\end{cases}
\]
\textbf{Решением СЛУ} является такой набор значений неизвестных, который является решением каждого уравнения в СЛУ.\\
\\
CЛУ называется \textbf{совместной} если она имеет хотя бы одно решение. 
В противном случае СЛУ называется \textbf{несовместной}.\\
\section{Эквивалентные системы линейных уравнений}
Две СЛУ от \underline{одних и тех же переменных} называются \textbf{эквивалентыми} если у них совпадают множества решений.
\section{Расширенная матрица линейных уравнений}
\[
(*) = 
\begin{cases}
  a_{11}x_1 + a_{12}x_2 + \cdots + a_{1n}x_n = b_1\\
  a_{21}x_1 + a_{22}x_2 + \cdots + a_{2n}x_n = b_2\\
  \cdots\cdots\cdots\cdots\cdots\cdots\cdots\cdots\cdots\cdots\cdots\\
  a_{m1}x_1 + a_{m2}x_2 + \cdots + a_{mn}x_n = b_m\\
\end{cases}
\]
\textbf{Расширенной матрицей} СЛУ (\textasteriskcentered) называется матрица вида 
$\begin{pmatrix}A \textbar b\end{pmatrix} = 
\begin{pmatrix}
\begin{tabular}{1111|1}
a_{11} & a_{12} & \cdots & a_{1n} & b_1\\
a_{21} & a_{22} & \cdots & a_{2n} & b_2\\
\vdots & \vdots & \vdots & \vdots & \vdots\\
a_{m1} & a_{m2} & \cdots & a_{mn} & b_m\\
\end{tabular}
\end{pmatrix}$,
где $A$ \textendash матрица коэффициентов при неизвестных, 
а $b$ \textendash вектор-слобец правых частей каждого уравнения СЛУ (\textasteriskcentered).
\section{Элементарные преобразования строк матрицы}
\textbf{Элементарными преобразованиями} называют следующие три преобразрования, меняющие вид матрицы:\\
\[
\begin{tabular}{1|1|1}
1 тип & К i-ой строке матрицы прибавить j-ую, умноженную на \lambda & \CYREREV_1{(i, j, \lambda)}\\
2 тип & Поменять местами i-ую и j-ую строки местами & \CYREREV_2{(i, j)}\\
3 тип & i-ую строку матрицы умножить на \underline{ненулевую} \lambda & \CYREREV_3{(i, \lambda)}, \lambda \neq 0\\
\end{tabular}
\]
\section{Ступенчатый вид матрицы}

Строка $(a_1, a_2, \cdots, a_i)$ называется \textbf{нулевой}, если $a_1 = a_2 = \cdots = a_i = 0$, 
и \textbf{ненулевой} в обратном случае $(\exists i:  a_i \neq 0)$.\\
\\
\textbf{Ведущим элементом} называется первый ненулевой элемент нулевой строки.\\
\\
Матрица $A \in \Mnm$ называется \textbf{ступенчатой} или имеет \textbf{ступенчатый вид}, если:\\
1) Номера ведущих элементов строго возрастают.\\
2) Все нулевые строки расположены в конце.
\[
\begin{pmatrix}
 0 & \heartsuit & * & * & * & \cdots & * & *\\
 0 & 0 & 0 & \heartsuit & * & * & \cdots & *\\
 0 & 0 & 0 & 0 & \heartsuit & * & \cdots & *\\
 \vdots & \vdots & \vdots & \vdots & \vdots & \vdots & \vdots & \vdots\\
 0 & 0 & 0 & 0 & 0 &  \cdots & 0 & \heartsuit\\
 0 & 0 & 0 & 0 & 0 & 0 & \cdots & 0\\
 0 & 0 & 0 & 0 & 0 & 0 & \cdots & 0\\
\end{pmatrix}
\]
где $*$ \textendash что угодно, $\heartsuit \textendash \neq 0$
\section{Улучшеный ступенчатый вид матрицы}
Говорят, что матрица имеет \textbf{улучшенный (усиленный) ступенчатый вид}, если:\\
1) Она имеет ступенчатый вид.\\
2) Все ведущие элементы матрицы равны $1$.\\
\[
\begin{pmatrix}
 0 & 1 & * & 0 & 0 & \cdots & 0 & 0\\
 0 & 0 & 0 & 1 & 0 & 0 & \cdots & 0\\
 0 & 0 & 0 & 0 & 1 & * & \cdots & 0\\
 \vdots & \vdots & \vdots & \vdots & \vdots & \vdots & \vdots & \vdots\\
 0 & 0 & 0 & 0 & 0 &  \cdots & 0 & 1\\
 0 & 0 & 0 & 0 & 0 & 0 & \cdots & 0\\
 0 & 0 & 0 & 0 & 0 & 0 & \cdots & 0\\
\end{pmatrix}
\]
\section{Теорема о виде, к которому можно привести матрицу при помощи элементарных преобразований}
\textbf{Теорема 1.} Любую матрицу можно привести к ступенчатому виду.\\
\textbf{Доказательство:}\\
\section{Общее решение совместной системы линейных уравнений}
Общим решением совместной СЛУ является множество наборов значений неизвестных, в которых главные неизвестные выражены 
через свободные (линейные комбинации от свободных неизвестных).
\section{Сколько может быть решений у системы линейных уравнений с действительными коэффициентами}
Всякая СЛУ с действительными коэффициентами либо несовместна, либо имеет ровно одно решение, либо имеет бесконечно много решений.
\section{Однородная система линейных уравнений. Что можно сказать про её множество решений?}
Однородной системой линейных уравнений (ОСЛУ) называется такая СЛУ, в которой каждое уравнение в правой части имеет 0.
Расширенная матрица имеет вид $(A | 0)$.\\
Очевидно: вектор $x = (0, 0, 0, \cdots 0)$ является решением всякой ОСЛУ.\\
Всякая ОСЛУ имеет либо решение (нулевое), либо бесконечно много решений.\\
\section{Свойство однородной системы линейных уравнений, у которой число неизвестных больше числа уравнений}
Всякая ОСЛУ, у которой число неизвестных больше числа уравнений имеет ненулевое решение $\Rightarrow$ имеет бесконечно много решений.
\section{Связь между множеством решений совместной системы линейных уравнений
и множеством решений соответсвующей ей однородной системы}
Пусть дана СЛУ $(*) = Ax = b $, и ассоциированная с ней ОСЛУ $Ax = 0$.\\
Пусть $L$ - множество решений ОСЛУ, а $c$ - решение СЛУ $(*)$.\\
Обозначим множество решений СЛУ $(*)$ за $S$.\\
Тогда $S = \{c + l | l \in L\}$.\\
Т.е. если сложить решение ОСЛУ, с произвольным решением СЛУ $(*)$, то полученный вектор снова будет решением СЛУ $(*)$.
\section{Обратная матрица}
\textbf{Обратной матрицей} к матрице $A \in Mn$ называется такая квадратная матрица $B \in \Mn$, что:\\
$AB = BA = E$. Матрица $B$ обозначается как $A^{-1}$.
\section{Перестановки множества $\{1, 2, \cdots, n\}$}
\textbf{Перестановкой} множества $X = \{1, 2, \cdots, n\}$ называется упорядоченный набор $(i_1, i_2, \cdots, i_n)$, 
в котором каждое число от 1 до n встречается ровно один раз.\\
\\
\textbf{Подстановка} (перестановка) из n элементов - это биективное отображение множества $\{1, 2, \cdots, n\}$ в себя. 
Обозначается: $
\begin{pmatrix}
 1 & 2 & 3 & \cdots & n\\
 i_1 & i_2 & i_3 & \cdots & i_n\\
\end{pmatrix}
$.
\section{Инверсия в перестановке. Знак перестановки. Чётные и нечётный перестановки.}
Говорят, что неупорядоченная пара ${i, j}$ образует \textbf{инверсию} в $\sigma$, если числа $i - j$ и $\sigma(i) - \sigma(j)$ имеют
разный знак, т.е. либо $i > j$ и $\sigma(i) < \sigma(j)$, либо $i < j$ и $\sigma(i) > \sigma(j)$.\\
\\
\textbf{Знаком (сигнатурой)} подстановки $\sigma$ называется число $sgn\ \sigma$,
такое что $sgn\ \sigma = (-1)^{inv\ \sigma}$, где $inv\ \sigma$ - число инверсий.
Знак может принимать значения $1$ и $-1$.\\
\\
Подстановка называется \textbf{чётной}, если её знак равен $1$, и \textbf{нечётной}, если её знак равен $-1$.
\section{Произведение двух перестановок}
Пусть даны две подстановки $\sigma$ и $\tau$ \in $S_n$. \textbf{Произведением} или (композицией) двух подстановок называется
такая подстановка $\sigma\tau$, что $(\sigma\tau)(i) = \sigma(\tau(i)), \forall i \in \{1, 2, \cdots, n\}$
\section{Тождественная перестановка и её свойства. Обратная перестановка и её свойства.}
\textbf{Тождественной (единичной)} подстановкой называется подстановка вида $
\begin{pmatrix}s
 1 & 2 & 3 & \cdots & n\\
 1 & 2 & 3 & \cdots & n\\
\end{pmatrix}
\in S_n$. Тождественная подстановка обозначается как $id$ (или $e$).\\
$id(i) = i, \forall i \in \{1, 2, \cdots, n\}$.\\
\textbf{Свойство:}\\
$id * \sigma = \sigma * id = \sigma, \forall \sigma \in S_n$\\
Пусть дана подстановка $\sigma = 
\begin{pmatrix}
 1 & 2 & 3 & \cdots & n\\
 \sigma(1) & \sigma(2) & \sigma(3) & \cdots & \sigma(n)\\
\end{pmatrix}$,
тогда \textbf{обратной подстановкой} к $\sigma$ называется подстановка
$\tau$, вида 
$
\begin{pmatrix}
 \sigma(1) & \sigma(2) & \sigma(3) & \cdots & \sigma(n)\\
 1 & 2 & 3 & \cdots & n\\
\end{pmatrix}
$, и обозначается, как $\sigma^{-1}$.\\
\textbf{Свойства:}\\
1) $\sigma^{-1}$ - единственная\\
2) $\sigma * \sigma^{-1} = \sigma^{-1} * \sigma = id$.\\
\section{Теорема о знаке произведения двух подстановок}
\textbf{Теорема:} Пусть даный $\sigma, \tau \in S_n$, тогда $sgn(\sigma\tau) = sgn(\sigma) * sgn(\tau)$\\
\section{Транспозиция. Знак транспозиции.}
Пусть дана подстановка $\tau \in S_n$,такая что $\tau(i) = j, \tau(j) = i, \tau(k) = k \forall k \neq i, j$. 
Такая подстановка $'tau$ называется \textbf{транспозицией}.\\
$sgn(\tau) = -1$\\
\section{Общая формула для определителя квадратной матрицы произвольного порядка}
Пусть дана матрица $A \in \Mn$, тогда\\
$|A| = \sum\limits_{\sigma \in S_n}(sgn\sigma)a_1_{\sigma(1)}a_2_{\sigma(2)}\cdots{a_n_{\sigma(n)}}$
\section{Определители 2-го и 3-го порядка}
\textbf{Определителем 2-го порядка} называется определитель квадратной матрицы $A \in \mathbb{R}^{2\times{2}} = \\
= \begin{pmatrix}
a & b\\
c & d\\
\end{pmatrix}
$.\\\\
$|A| = ad - bc$.\\
\textbf{Определителем 3-го порядка} называется определитель квадратной матрицы $A \in \mathbb{R}^{3\times{3}} =
\begin{pmatrix}
 a & b & c\\
 d & e & f\\
 g & h & k\\
\end{pmatrix}
$.\\\\
$|A| = aek + bjg + cdh - ceg - afh - bdk$\\
\section{Поведение определителя при разложении строки (столбца) в сумму двух}
Пусть дана квадратная матрица $A \in \mathbb{R}^{n\times{n}}$.\\\\
Тогда если $A_{(i)} = A_{(i)}^1 + A_{(i)}^2$, то $|A| = 
\begin{tabular}{|1|}
A_{(1)}\\
\cdots\\
A_{(i)}^1\\
\cdots\\
A_{(n)}\\
\end{tabular}
+ 
\begin{tabular}{|1|}
A_{(1)}\\
\cdots\\
A_{(i)}^2\\
\cdots\\
A_{(n)}\\
\end{tabular}$.\\\\
Аналогично если $A^{(i)} = A^{(j)}_1 + A^{(j)}_2$, то $|A| = |A^{(1)} \cdots A^{(j)}_1 \cdots A^{(n)}| + 
|A^{(1)} \cdots A^{(j)}_2 \cdots A^{(n)}|$
\section{Поведение определителя при перестановке двух строк (столбцов)} 
Элементарное преобразование второго типа, а именно перестановка двух строк (столбцов) местами
\textbf{меняет знак определителя} и \textbf{не меняет значение определителя}.
\section{Поведение определителя при прибавлению к строке (столбцу) другой строки (столбца), умноженного на скаляр}
Элементарное преобразование первого типа, а именно прибавление к строке (столбцу) матрицы другой строки
(столбца), умноженного на скаляр
\textbf{не меняет знак определителя} и \textbf{не меняет значение определителя}.
\section{Верхнетреугольный и нижнетреугольные матрицы}
\textbf{Верхнетреугольной матрицей} называется такая квадратная матрица $A \in \mathbb{R}^{n\times{n}}$,
у которой элементы, стоящие ниже главной диагонали равны нулю. Т.е. $a_{ij} = 0 \forall i,j = {0, \cdots, n}: i > j$.\\
\textbf{Нижнетреугольной матрицей} называется такая квадратная матрица $A \in \mathbb{R}^{n\times{n}}$,
у которой элементы, стоящие выше главной диагонали равны нулю. Т.е. $a_{ij} = 0 \forall i,j = {0, \cdots, n}: j > i$.\\
\section{Определитель верхнетреугольной (нижнетреугольной) матрицы}
Определитель верхнетреугольной матрицы равен определителю нижнетреугольной матрицы и \textbf{равен
произведению её элементов, стоящих на главной диагонали}.
\section{Определитель диагональной матрицы. Определитель единичной матрицы.}
Диагональную матрицу можно считать частным случаем как верхнетреугольной, так и нижнетреугольной матрицы,
и следовательно \textbf{определитель диагональной матрицы} равен произведению её элементов, стоящих
на главной диагонали.\\
\textbf{Определитель единичной матрицы}, которая является частным случаем диагональной матрицы, по той же логике
равен 1.
\section{Матрица с углом нулей и её определитель}
\textbf{Матрицей с углом нулей} называется квадратная матрицы $A \in \mathbb{R}^{n\times{n}}$ вида
$A = 
\begin{pmatrix}
P & Q\\
0 & R\\
\end{pmatrix}
$ или $A = 
\begin{pmatrix}
P & 0\\
Q & R\\
\end{pmatrix}
$, где $P \in \mathbb{R}^{k\times{k}}$, $R \in \mathbb{R}^{(n-k)\times{(n-k)}}$.\\
\textbf{$\det A$} = $\det P\det R$.\\
\section{Определитель произведения двух матриц.}
Пусть даны две квадратные матрицы одного порядка $A, B \in \mathbb{R}^{n\times{n}}$. Тогда
$|AB| = |A| * |B|$
\section{Дополнительный минор к элементу квадратной матрицы}
Пусть дана квадратная матрица $A \in \mathbb{R}^{n\times{n}}$. \textbf{Дополнительным минором} к $a_{ij}$ называется
определитель матрицы порядка $(n - 1)$, получаемой удалением из исходной матрицы i-ой строки и j-ого столбца.
Обозначается \overline{$M_{ij}$}.
\section{Алгебраическое дополнение к элементу квадратной матрицы}
Пусть дана квадратная матрица $A \in \mathbb{R}^{n\times{n}}$. \textbf{Алгебраическим дополнением} к $a_{ij}$ называется
число $A_{ij} = (-1)^{i + j}$\overline{$M_{ij}$}.
\section{Формула разложения определителя по строке (столбцу)}
Пусть дана квадратная матрица $A \in \mathbb{R}^{n\times{n}}$. Тогда для любого фиксированного $j \in \{1, \cdots, n\}$
$|A| = a_{1j}A_{1j} + a_{2j}A_{2j} + \cdots + a_{nj}A_{nj} = \sum\limits_{i = 1}^n{a_{ij}A{ij}}$\\
Аналогично для любой фиксированной строки.
\section{Лемма о фальшивом разложении определителя}
Пусть дана квадратная матрица $A \in \mathbb{R}^{n\times{n}}$. Тогда при любом $i, k \in \{1, \cdots, n\}, i \neq k$:
$\sum\limits_{j = 1}^n{a_{ij}A_{ik}} = 0$.\\
Аналогично для столбцов.
\section{Невырожденная матрица}
Пусть дана квадратная матрица $A \in \mathbb{R}^{n\times{n}}$. Тогда $A$ называется \textbf{невырожденной} $\Leftrightarrow$
$|A| \neq 0$, и \textbf{вырожденной} в противном случае.
\section{Присоединённая матрица}
Пусть дана квадратная матрица $A \in \mathbb{R}^{n\times{n}}$.\textbf{Присоединённой матрицей} к $A$ называется
матрица $\hat{A} = (A_{ij})^T$.
\[
\hat{A} = 
\begin{pmatrix}
A_{11} & A_{21} & \cdots & A_{n1}\\
A_{12} & A_{22} & \cdots & A_{n2}\\
\vdots & \vdots & \vdots & \vdots\\
A_{1n} & A_{2n} & \cdots & A_{nn}\\
\end{pmatrix}
\]
\section{Критерий обратимости квадратной матрицы}
Пусть дана квадратная матрица $A \in \mathbb{R}^{n\times{n}}$. Тогда $A$ является обратимой $\Leftrightarrow$ $|A| \neq 0$.
\section{Явная формула для обратной матрицы}
Пусть дана квадратная матрица $A \in \mathbb{R}^{n\times{n}}$. Тогда матрица $B \in \mathbb{R}^{n\times{n}}$ называется
\textbf{обратной к $A$} $\Leftrightarrow$ $A$ - обратима. При этом $B = \frac{1}{|A|}\hat{A}$. Обозначается $A^{-1}$.
\section{Критерий обратимости произведения двух матриц. Матрица, обратная к произведению двух матриц.}
Пусть даны две квадратные матрицы $A, B \in \mathbb{R}^{n\times{n}}$.\\
Тогда матрица $AB$ обратима тогда и только тогда, когда $A$ обратима и $B$ обратима.\\
Причём $(AB)^{-1} = B^{-1}A^{-1}$.\\
\section{Формулы Крамера}
Пусть дана СЛУ $Ax = b$, где $A \in \mathbb{R}^{n\times{n}}$, а $x \in \mathbb{R}^n$ - столбец неизвестных.\\
Если $|A| \neq 0$, то единственное решение СЛУ можно получить по формулам $x_i = \frac{|A_i|}{|A|}$, где
$\forall i = 1, 2, \cdots, n A_i$ - матрица, полученная заменой $i$-ого столбца матрицы $A$ на столбец $b$.
\section{Что такое поле?}
\textbf{Полем} называется множество $\mathbb{F}$, на котором определены две операции:\\
1) Сложение: $(a, b) \longrightarrow a + b$\\
2) Умножение: $(a, b) \longrightarrow ab$\\
Причём $\forall a, b, c \in \mathbb{F}$ выполняются следующие аксиомы:
1) $a + b = b + a$\\
2) $a + (b + c) = (a + b) + c$\\
3) $\exists 0: a + 0 = a$\\
4) $\exists -a: a + (-a) = 0$\\
5) $(a + b)c = ac + bc$\\
6) $ab = ba$\\
7) $a(bc) = (ab)c$\\
8) $\exists 1: a * 1 = a$\\
9) $\exists a^{-1}: a * a^{-1} = 1$\\
\\
\section{Алгебраическая форма комплексного числа. Сложение, умножение и деление комплексных чисел в алгебраической форме.}
Комлексное число $z \in \mathbb{C}$, представленное в виде $z = a + bi$, где $a, b \in \mathbb{R}$, а $i^2 = -1$,
причём $a$ называется действительной частью, числа $z$, а $b$ называется мнимой часть.\\
\section{Комплексное сопряжение и его свойства. Сопряжение суммы и произведения двух комплексных чисел}
Пусть дано комплексное число $z = a + bi$, тогда комплексное число вида \overline{$z$}$ = a - bi$ 
называется его \textbf{комплексным сопряжением}.\\
Свойства: $\forall z, w \in \mathbb{C}$\\
1) $z$\overline{$z$}$ = a^2 + b^2 \in \mathbb{R}$\\
2) \overline{\overline{$z$}} = $z$\\
3) \overline{$z + w$} = \overline{$z$} + \overline{$w$}\\
4) \overline{$z * w$} = \overline{$z$} * \overline{$w$}\\
\section{Геометрическая модель комплексных чисел. Интерпретация в ней сложения и сопряжения}
Пусть даны комплексные числа $z = a + bi$.\\
Его можно воспринимать как точку (а лучше вектор) на плоскости $\mathbb{R}^2$ с координатами $(a, b)$.\\
Сумму $z + w, \forall z, w \in \mathbb{C}$ можно воспринимать как сумму соответсвующих векторов, а
комплексное сопряжение к $z$ равносильно вектору, отражённому относительно действительной оси.
\section{Модуль комплексного числа и его свойства: неотрицательность, неравенство треугольника, модуль произведения
	      двух комплексных чисел}
Пусть дано комплесное число $z = a + bi \in \mathbb{C}$\\
Тогда модулем $z$ называется число $|z|$, такое что $|z| = \sqrt{a^2 + b^2}$\\
Свойства: $\forall z, w \in \mathbb{C}$\\
1) $|z| \geqslant 0$, причём $|z| = 0 \Leftrightarrow z = 0$\\
2) $|z + w| \leqslant |z| + |w|$\\
3) $|zw| = |z||w|$\\
Комплексное число можно так же предстваить в виде: $z = a + bi = |z|(\frac{a}{|z|} + \frac{b}{z}i)$\\
\section{Аргумент комплексного числа}
Пусть дано комплексное число $z = a + bi \neq 0$.\\
Тогда аргументом комплексного числа $z$ называется такое число $\varphi \in \mathbb{R}$, что
$\cos{\varphi} = \frac{a}{|z|}$, а $\sin{\varphi} = \frac{b}{|z|}$.\\
В геометрической модели аргумент это угол между осью абсцисс и вектором $z$.
\section{Тригонометрическая форма комлплексного числа. Умножение и деление комплексных чисел в тригономестрической
	      форме}
\textbf{Тригономестрической формой} комплексного числа $z$ называется его предстваление в виде 
$z = |z|(\cos{\varphi} + i\sin{\varphi})$.\\
Пусть даны два комплексных числа $z_1, z_2$, тогда\\
\textbf{Произведением} двух комплексных чисел $z_1$ и $z_2$ называется такое число $w \in \mathbb{C}$, что
$w = z_1z_2 = |z_1||z_2|(\cos{\varphi_1 + \varphi_2} + i\sin{\varphi_1 + /varphi_2})$.\\
\textbf{Произведением} двух комплексных чисел $z_1$ и $z_2$ называется такое число $w \in \mathbb{C}$, что
$w = \frac{z_1}{z_2} = \frac{|z_1|}{|z_2|}(\cos{\varphi_1 - \varphi_2} + i\sin{\varphi_1 - /varphi_2})$.\\
\section{Формула Муавра}
Пусть дано комплексное число $z = |z|(\cos{\varphi} + i\sin{\varphi})$.\\
Тогда $z^n = |z|^n(\cos{n\varphi} + i\sin{n\varphi})$
\section{Извлечение корней из комплексного числа}
Пусть дано комплексное число $z \in \mathbb{C}$ и $n \in \mathbb{R}$.\\
Тогда корнем n-ой степени из числа $z$ называется такое число $w \in \mathbb{C}$, что $w^n = z$.\\
$\sqrt[n]{z} = \{w \in \mathbb{C} | w^n = z\}$\\
Если $z = 0$, то $|z| = 0 \Rightarrow |w| = 0 \Rightarrow w = 0 \Rightarrow \sqrt[n]{0} = \{0\}$.\\
Если $z \neq 0$, то:\\
$z = |z|(\cos{\varphi} + i\sin{\varphi})$\\
$w = |w|(\cos{\psi} + i\sin{\psi})$\\
$z = w^n = |w|^n(\cos{(n\psi)} + i\sin{(n\psi)})$\\
$z = w^n \Leftrightarrow 
\begin{cases}
|w| = \sqrt[n]{|z|}\\
\psi = \frac{\varphi + 2\pi{k}}{n}, & k \in \{0, 1, 2, \cdots n - 1\}\\
\end{cases}
$
\section{Основная теорема алгебры комплексных чисел}
\section{Теорема Безу и её следствие}
\section{Кратность корня многочлена}
\section{Векторное пространство}
\textbf{Векторным пространством} над полем $F$ нахывается такое множество $V$, на котором определены две операции:\\
1)Сложение: $\forall a, b \in V: a + b$\\
2)Умножение на скаляр: $\forall a \in V, \lambda \in F: \lambda{a} $\\
Причём $\forall a, b, c \in V$ и $\forall v, u \in F$ выполняются следующие свойства (аксиомы векторного пространства):\\
1)$a + b = b+ a$\\
2)$a + (b +c) = (a + b) + c $\\
3)$\exists \overline{0}: a + 0 = a$\\
4)$\exists -a: a + (-a) = 0$
5)$(v + u)a = va + ua$\\
6)$(a + b)u = au + bu$\\
7)$a(vu) = (av)u$\\
8)$\exists 1: a * 1 = a $\\
\section{Подпространство векторного пространства}
Пусть $V$ - векторное пространство над $F$.\\
Тогда подмножество $U$ множества $V$ называется подпространством, если:\\
1) $0 \in U$ (Очень важное условие, оно гарантирует, что множество непусто)\\ 
2) $\forall x, y \in U: x + y \in U$\\
3) $\forall x \in U, a \in F: ax \in U$\\
\section{Линейная комбинация конечного набора векторов линейного пространства}
Пусть $V$ - векторное пространство над $F$, и даны $v_1, v_2, \cdots v_n \in V$\\
Тогда линейной комбинацией набора векторов называется выражение вида 
$a_1v_1 + a_2v_2 + \cdots + a_nv_n$, где $a_1, a_2, \cdots a_n \in F$.
\section{Линейная оболочка подмножества векторного пространства}
Пусть $V$ - векторное пространство над $F$, а $S$ - подпространство в $V$.\\
Тогда линейной оболочкой $S$ называется множество всех возможных линейных комбинаций
векторов из $S$.
\section{Две общих конструкции подпространств в пространстве $F^n$}
Пусть дана ОСЛУ $Ax = 0$.\\
Тогда множество решений этой ОСЛУ является подпространством в $F^n$.\\
Пусть дано $S \subseteq V$, тогда $\langle S \rangle$ - подпространство в $F^n$.
\section{Линейная зависимость конечного набора векторов}
Пусть $V$ - векторное пространство над $F$, и даны $v_1, v_2, \cdots v_n \in V$\\
Тогда система векторов $\{ v_1, v_2, \cdots, v_n \}$ называется линейно зависимой,
если существует их нетривиальная линейная комбинация, равная нулю.\\
Т.е. $\exists (a_1, \cdots, a_n) \neq (0 \cdots 0): a_1v_1 + \cdots + a_nv_n = 0$\\
\section{Линейная независимость конечного набора векторов}
Пусть $V$ - векторное пространство над $F$, и даны $v_1, v_2, \cdots v_n \in V$\\
Тогда система векторов $\{ v_1, v_2, \cdots, v_n \}$ называется линейно независимой,
если не существует их нетривиальной линейной комбинации, равной нулю.\\
Т.е. $\nexists (a_1, \cdots, a_n) \neq (0 \cdots 0): a_1v_1 + \cdots + a_nv_n = 0 \Rightarrow a_1 = \cdots = a_n = 0$\\
\section{Критерий линейной зависимости конечного набора векторов}
Пусть $V$ - векторное пространство над $F$, и даны $v_1, v_2, \cdots v_n \in V$\\
Тогда $v_1, v_2, \cdots, v_n \in V$ линейно зависимы, если $\exists i \in \{1, 2, \cdots, n\}$, такой что
$v_i$ является линейной комбинацией остальных векторов.\\
Формально: $a_1v_1 + a_2v_2 + \cdots + a_{i-1}v_{i-1} + a_{i+1}v_{i+1} + \cdots + a_nv_n = v_i$.
\section{Основная лемма о линейной зависимости}
\section{Базис векторного пространства}
Пусть $V$ - векторное пространство над $F$.\\
Тогда система векторов $S \subseteq V$ называется базисом пространства $V$, если:\\
1) $S$ линейно независима\\
2) $\langle{S}\rangle = V$\\
\section{Конечномерные и бесконечномерные векторные пространства}
Векторное пространство $V$ является конечномерным, если имеет конечный базис, и бесконечномерным иначе.
\section{Размерность конечномерного векторного пространства}
Размерностью конечномерного вектрорного пространства $V$ называется число $dim V \in \mathbb{R}$ равное количеству
векторов в базисе $V$.
\section{Характеризация базисов конечномерного векторного пространства в терминах единственности линейного выражения
векторов}
Пусть $V$ - векторное пространство над $F$, а $e_1 \cdots e_n$ - базис пространства $V$. 
Тогда $\forall v \in V$ единственным образом представим в виде $v = x_1e_1 + \cdots + x_ne_n$, где $x_i \in F$.
\section{ФСР ОСЛУ}
Пусть дана ОСЛУ $Ax = 0$.\\
Тогда множество решений этой ОСЛУ задаёт векторное пространство $S$. 
Тогда фундаментальной системой решений ОСЛУ называется произвольный базис в $S$.
\section{Лемма о добавлении вектора к конечной, линейно независимой системе}
Пусть дана линейно независимая система векторов $v_1, \cdots, v_n \in V$, и вектор $v \in V$.\\
Тогда, при добавлении вектора $v$ в систему:\\
1) Система $v_1, \cdots, v_n, v$ линейно независима\\
2) $v \in \langle v_1, \cdots, v_n \rangle$
\end{document}