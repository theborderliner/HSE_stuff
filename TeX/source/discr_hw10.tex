\documentclass[a4paper,11pt]{report}
\usepackage[T2A]{fontenc}
\usepackage[utf8]{inputenc}
\usepackage[english,russian]{babel}
\usepackage{amsmath} %математические пакеты
\usepackage{amsfonts}
\usepackage{amssymb}
\usepackage[pdftex,unicode]{hyperref}
\usepackage[top=2cm,
left=2cm,
right=2cm,
bottom=2cm]{geometry}
\usepackage[dvipsnames]{xcolor}

\title{Коллок по линалу}
\author{Пешехонов Иван. БПМИ1912}
\date{\today}
\begin{document}
\section{№ 1.}
a) $R = \{(1, 1), (2, 2), (3, 3), (1, 2), (1, 3), (3, 2)\}$\\
Очевидно рефлексивно, не симметрично (нет пары $(2, 1)$), транзитивно.\\
Итог: не эквивалентно.\\
б) $R = \{(1, 1), (1, 2), (2, 1), (2, 2)\}$\\
Рефлексивно ($(1, 1), (2, 2)$), симметрично ($(1, 2) \Leftrightarrow (2, 1)$), транзитивно.\\
Итог: эквивалентно.
\section{№ 2.}
\section{№ 4.}
а) Всего пар можнет быть $6 * 6 = 36$ (6 цифр на первое место, и, т.к. допускаются повторения, то 6 цифр на второе).
По условию имеются 33 пары, т.е. из всех вариантов есть 3 пары, которые не попали в множество. Допустим, что это
например $(5, 5), (2, 5)$ и $(5, 2)$, тогда остальные пары точно будут симметричны.\\
б) 
\end{document}
