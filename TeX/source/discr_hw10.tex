\documentclass[a4paper,11pt]{report}
\usepackage[T2A]{fontenc}
\usepackage[utf8]{inputenc}
\usepackage[english,russian]{babel}
\usepackage{amsmath} %математические пакеты
\usepackage{amsfonts}
\usepackage{amssymb}
\usepackage[pdftex,unicode]{hyperref}
\usepackage[top=2cm,
left=2cm,
right=2cm,
bottom=2cm]{geometry}
\usepackage[dvipsnames]{xcolor}

\title{Дискра ДЗ.}
\author{Пешехонов Иван. БПМИ1912}
\date{\today}
\begin{document}
\chapter{Домашнее задание № 10.}
\section{№ 1.}
a) $R = \{(1, 1), (2, 2), (3, 3), (1, 2), (1, 3), (3, 2)\}$\\
Очевидно рефлексивно, не симметрично (нет пары $(2, 1)$), транзитивно.\\
Итог: не эквивалентно.\\
б) $R = \{(1, 1), (1, 2), (2, 1), (2, 2)\}$\\
Рефлексивно ($(1, 1), (2, 2)$), симметрично ($(1, 2) \Leftrightarrow (2, 1)$), транзитивно.\\
Итог: эквивалентно.
\section{№ 2.}
Чего от меня тут хотят?
\section{№ 3.}
1) Ну, очевидно, что \overline{$P_1$} - не рефлексивно, поэтому транзитивность не будет выполняться например на таком
примере: $x$\overline{$P_1$}$y$, $y$\overline{$P_1$}$x \nRightarrow x$\overline{$P_1$}$x$.\\
2) Возьмём две пары элементов из каждого подмножества:\\
$(a, b) \in P_1$\\
$(b, c) \in P_2$\\
Рассмотрим транзитивность: $(a, b);(b, c) \nRightarrow (a, c)$, потому что $c$, например, не лежит в $P_1$, а значит
$(a, c) \notin P_1 \cap P_2$.\\
3) Тут по сути отрицание предыдущего пункта: транзитивность будет выполняться, потому что мы так же выберем две
пары, и увидим, что пара элементов, которые лежат в разных подмножествах, принадлежит их объединению.\\
\section{№ 4.}
а) Всего пар можнет быть $6 * 6 = 36$ (6 цифр на первое место, и, т.к. допускаются повторения, то 6 цифр на второе).
По условию имеются 33 пары, т.е. из всех вариантов есть 3 пары, которые не попали в множество. Допустим, что это
например $(5, 5), (2, 5)$ и $(5, 2)$, тогда остальные пары точно будут симметричны.\\
б) Зафиксируем цифру $a$, тогда есть 6 пар, где $a$ идёт на первом месте, и столько же пар, если $a$ на последнем месте,
и транзитивность выполняется: $xRa, aRy \Leftrightarrow xRy$.\\ 
Но может случиться так, что в наборе не будет трёх пар, таких что в них присутсвует $a$. Тогда из 33 пар, для одной
не будет выполняться транзитивность,\\
\section{№ 5.}
a) Очевидно, рефлексивно. Если $xRy$, то $yRx$ (если у $x$ и $y$ одинаковая последняя цифра, то у $y$ и $x$ одинаковая последняя цифра 0\_o),
значит симметрично. Если у $x$ и $y$ одинаковая последняя цифра, и у $y$ и $z$ та же фигня, значит и у $x$ и $y$ одинаковая последняя цифра.\\
Вывод: эквивалентно.\\
б) Уже по формулировке не рефлексивно ($x$ не отличается от $x$ на одну цифру).\\
Вывод: не эквивалентно.\\
в) Очевидно, рефлексивно (разница между суммами цифр $x$ и $y$ равна $0$).\\
Пусть известно, что $S_x - S_y$ - чётно, проверим $S_y - S_x$. Возьмём по абсолютной величине, т.к. знак в действительности нас
мало волнует: $|S_y - S_x| = |S_x - S_y|$ - чётно. Значит, симметрично.\\
Ну и транзитивно кст, если $S_x - S_y$ - чётно, и $S_y - S_z$ -  чётно, то $S_x - S_z = S_x - (S_y - G) = (S_x - S_y) + G$ - чётно,
где $G = (S_y - S_z)$ - какое-то чётное число.\\
Вывод: эквивалентно.
\section{№ 6.}
Переберём все варианты:\\
Всего на множестве $\{1, 2, 3, 4\}$ можно составить четыре пары, для которых выполняется рефлексивность: 
$(1, 1), (2, 2), (3, 3), (4, 4)$\\
Теперь посчитаем количество симместричных пар, оно будет равно $2 * 
\begin{pmatrix}
4\\
2
\end{pmatrix}
= 2 *6 = 12$.\\
И для всех этих пар уже будет выполняться транзитивность, значит всего отноешний эквивалентности будет
$12 + 4 = 16$.
\section{№ 9.}
Утверждение: $f$ - биекция. На самом деле это логично: предположим, что при применении к множеству $A$ отображения $f$
какие-то два элемента перешли в один. Тогда второй раз отображение уже не применить, потому что получится что один элемент 
переходит в два разных. По этой же причине не один элемент не может остаться свободным, т.к. если хотя бы один не занят, значит
какой-то другой "переполнен". Будем считать, что доказано.\\
Значит имеем биективное отображение, переводящее множество в себя. Это по определению подстановка, обозначим её $\sigma$.
Имеем произведение двух подстановок на семи элементов $\sigma, \tau \in S_7$, такое что $\sigma\tau = id$.
Есть три варианта, когда такое возможно: $\sigma = \tau = id$. (Круто, одно отображение уже нашли).\\
2) $\tau = \sigma^{-1}$, тогда $\sigma\sigma^{-1} = id$, но этот вариант там не подходит, потому что он предполагает, что у нас
применяются разные отображения, в то время как по условию оно одно. Из этого вытекает третий способ:\\
3) $\sigma^2 = id$. Чтобы такое равенство выполнялось, необходимо, чтобы подстановка $\sigma$ раскладывалась в произведение
транспозиций.\\
Найдём теперь возможное количество транспозиций:\\
Ну во-первых их может вообще не быть, тогда все элементы в произведении подстановок остаюстя на своих местах, т.е. 
$\sigma = id$, этот способ мы уже нашли.\\
Может быть всего одна транспозиция, тогда нам нужно найти количество способов выбрать 2 элемента из 7.
Таких способов, очевидно $
\begin{pmatrix}
7\\
2
\end{pmatrix}
 = 21$.\\
Может быть две транспозиции, тут немного извратимся: посчитаем количество способов выбрать 3 элемента, которые не попадут в транспозицию
($\begin{pmatrix}
   7\\
   3
  \end{pmatrix}
$) и умножим это дело на число пар, которые можно составить из четырёх элементов (фиксируем первый элемент первой пары,
тогда есть 3 варианта выбрать второй элемент, и когда мы таким образом выбрали первую пару, то вторая определится однозначно, имеем 3 способа).
Т.е. всего $3 * 
\begin{pmatrix}
 7\\
 3
\end{pmatrix}
 = 105$.\\
А может быть и три транспозиции, тут мы просто по очереди фиксируем каждый из семи возможных элементов, которые останутся на своём месте,
и вычисляем количство способов составить транспозиции из оставшихся семи элементов: $7 * 
\begin{pmatrix}
6\\
2
\end{pmatrix}
 = 105$.\\
А теперь всё просуммируем: $1 + 21 + 105 + 105 = 232$.\\
Ответ: $232$. :D
\end{document}
