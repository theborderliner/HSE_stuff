\documentclass[a4paper,11pt]{report}

\usepackage[T2A]{fontenc}
\usepackage[utf8]{inputenc}
\usepackage[english,russian]{babel}
\usepackage{amsmath} %математические пакеты
\usepackage{amsfonts}
\usepackage{amssymb}
\usepackage[pdftex,unicode]{hyperref}
\usepackage[dvipsnames]{xcolor}

\usepackage[top=2cm,
left=2cm,
right=2cm,
bottom=2cm]{geometry}

\title{Дискретная математика. Коллоквиум.}
\author{Пешехонов Иван. БПМИ1912}
\date{\today}

\begin{document}
\maketitle
\tableofcontents
\chapter{Список определений.}
\section{Логические операции: конъюнкция, дизъюнкция и отрицание}
\textbf{Логическими операциями (функциями)} называются функции, которые зависят
от набора \textbf{логических переменных}, принимающих значения 0 или 1, и сами
так же принимают значения 0 (ложь) или 1 (истина).\\
\\
Пусть есть две логические переменные: $A$ и $B$. Тогда операции отрицание, 
конъюнкция и дизъюнкция задаются следующей \textbf{таблицей истинности}:\\
\[
\begin{tabular}{1|1|1|1|1}
A & B & \={A} & A \wedge\ B & A \vee\ B\\
0 & 0 & 1 &  0 & 0\\
0 & 1 & 1 & 0 & 1\\
1 & 0 & 0 & 0 & 1\\
1 & 1 & 0 & 1 & 1\\
\end{tabular}
\]
\section{Логические операции: импликация, XOR (исключающее или) и эквивалентность}
Пусть есть две логические переменные: $A$ и $B$. Тогда операции эквивалентность, 
импликация и XOR задаются следующей таблицей истинности:\\
\[
\begin{tabular}{1|1|1|1|1}
A & B & A \leftrightarrow B & A \rightarrow B & A \bigoplus B\\
0 & 0 & 1 & 1 & 0\\
0 & 1 & 0 & 1 & 1\\
1 & 0 & 0 & 0 & 1\\
1 & 1 & 1 & 1 & 0\\
\end{tabular}
\]
\section{Булевы функции. Задание таблицей истинности и вектором значений}
\textbf{Булевыми (логическими) функциями} называются функции, которые зависят
от набора \textbf{логических переменных}, принимающих значения 0 или 1, и сами
так же принимают значения 0 (ложь) или 1 (истина).\\
Есть два основных способа задать булеву функцию:\\
\subsection{Задание булевой функции через таблицу истинности}
Если булева функция зависит от $k$ переменных, то первые $k$ столбцов
таблицы соотвествуют переменным, а $k + 1$-ый столбец соотвествует
значению функции на соотвествующем наборе значений. Кроме того наборы
значений переменных вычисляются по следующему правилу: $i$-ая строка таблицы
содержит двоичную запись числа $i - 1$. Таким образом таблица истинности содержит
$2^k$ строк.
\subsection{Задание булевой функции через вектор значений}
Любую функцию от $k$ переменных можно задать столбцом её значений
в таблице истинности, причём значений в векторе будет ровно $2^k$. 
Пример:\\
Булевой функции, заданой вектором значений $f(a, b) = 0001$ соотвествует
таблица истинности \[
\begin{tabular}{1|1|1}
a & b & f(a, b)\\
0 & 0 & 0\\
0 & 1 & 0\\
1 & 0 & 0\\
1 & 1 & 1\\
\end{tabular}
\]
\section{Существенные и фиктивные переменные булевой функции}
Пусть задана булевая функция от $k$ переменных: $f(x_1, x_2, \cdots, x_k)$.
Переменная $x_i$ называется \textbf{фиктивной}, если выполняется 
тождество\\
$f(x_1, x_2, \cdots, x_{i - 1}, x_i, x_{i + 1}, \cdots, x_k) =
f(x_1, x_2, \cdots, x_{i - 1}, 0, x_{i + 1}, \cdots, x_k)$.\\
Если же для переменной $x_i$ такое тождество не выполняется,
то говорят, что переменная $x_i$ является \textbf{существенной}.
\section{Множество, подмножество, равенство множеств}
\textbf{Множеством} принято называть совокупность уникальных объектов
без учёта отношений между этими объектами.\\
Пусть дано множество $A = {1, 2, 3, 4, 5}$. Говорят, что множество
$B = {2, 4, 5}$ является \textbf{подмножеством} $A$, если $B$ входит в $A$,
т.е. $\forall e \in B: e \in A$. Обозначается как $B \subseteq A$.
Говорят, что два множества $A$ и $B$ \textbf{равны}, если $A \sebsetqe B$ и 
$B \subseteq A$.
\section{Операции с множествами: объединение, пересечение, разность, 
симметрическая разность. Диаграммы Эйлера-Венна.}
Пусть заданы два множества $A$ и $B$.\\
1) \textbf{Объединением} множеств $A$ и $B$ является множество $C$, такое что
$C = {a : (a \in A) \vee (a \in B)}$. Обозначается $A \cup B = C$.\\
1) \textbf{Пересечением} множеств $A$ и $B$ является множество $C$, такое что
$C = {a : (a \in A) \wedge (a \in B)}$. Обозначается $A \cap B = C$.\\
1) \textbf{Разностью} множеств $A$ и $B$ является множество $C$, такое что
$C = {a : (a \in A) \wedge (a \notin B)}$. Обозначается $A \textbackslash B = C$.\\
1) \textbf{Симметрической разностью} множеств $A$ и $B$ является множество $C$, такое что
$C = (A \cup B)\textbackslash(A \cap B)$. Обозначается $A \Delta B = C$.\\
\end{document}          
