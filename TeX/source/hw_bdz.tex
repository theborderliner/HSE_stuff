\documentclass[a4paper,12pt]{report}

\usepackage[T2A]{fontenc}
\usepackage[utf8]{inputenc}
\usepackage[english,russian]{babel}
\usepackage{amsmath} %математические пакеты
\usepackage{amsfonts}
\usepackage{amssymb}
\usepackage[pdftex,unicode]{hyperref}
\usepackage[dvipsnames]{xcolor}
\usepackage{graphicx}

\usepackage[top=2cm,
left=2cm,
right=2cm,
bottom=2cm]{geometry}

\title{Большое домашнее задание. Математический анализ.}
\author{Пешехонов Иван. БПМИ1912}
\date{\today}

\begin{document}
\maketitle
\clearpage
\chapter{Вариант 24.}
\section{№ 1.}
$\lim\limits_{x \to \infty}\left(\frac{-3 + 2x}{1 - 4x}\right)^{\frac{3x^2 - 1}{x + 1}} =
\lim\limits_{x \to \infty}\left(1 + (\frac{-3 + 2x}{1 - 4x} - 1)\right)^{\frac{3x^2 - 1}{x + 1}} =
\lim\limits_{x \to \infty}\left(1 + \frac{-4 + 6x}{1 - 4x}\right)^{\frac{3x^2 - 1}{x + 1}} =
\lim\limits_{x \to \infty}\left(1 + \frac{1}{\frac{1 - 4x}{-4 + 6x}}\right)^{\frac{3x^2 - 1}{x + 1}} = \\
= \lim\limits_{x \to \infty}\left(1 + \frac{1}{\frac{1 - 4x}{-4 + 6x}}\right)^{\frac{-4 + 6x}{1 - 4x} * \frac{1 - 4x}{-4 + 6x} * \frac{3x^2 - 1}{x + 1}} =
e^{\lim\limits_{x \to \infty}\left(\frac{1 - 4x}{-4 + 6x} * \frac{3x^2 - 1}{x + 1}\right)} = 
e^{ \lim\limits_{x \to \infty}\left(\frac{3x^2 - 1 - 12x^3 + 4x}{-4x - 4 +6x^2 + 6x}\right)} =
e^{ \lim\limits_{x \to \infty}\left(\frac{\frac{3x^2 - 1 - 12x^3 + 4x}{x^3}}{\frac{-4 +6x^2 + 2x}{x^3}}\right)} = \\\\
= e^{\lim\limits_{x \to \infty}\left(\frac{-12}{\frac{6}{x}}\right)} =
e^{\lim\limits_{x \to \infty}- 12x} = \lim\limits_{x \to \infty}e^{-\infty} = 0$
\section{№ 2.}
$\lim\limits_{x \to \frac{\pi}{3}}\left(4 + 3\cos{3x}\right)^{\frac{1}{\tg{3x}}} =
\lim\limits_{x \to \frac{\pi}{3}}\left(1 + (3 + 3\cos{3x})\right)^{\frac{1}{\tg{3x}}}$\\\\
Проведём замену: $t = x - \frac{\pi}{3}$, тогда $x = t + \frac{\pi}{3}$:\\\\
$\lim\limits_{t \to 0}\left(1 + (3 + 3\cos{(3t + \pi)})\right)^{\frac{1}{\tg{(3t + \pi)}}} = 
\lim\limits_{t \to 0}\left(1 + (3 - 3\cos{3t})\right)^{\frac{1}{3 - 3\cos{3t}} * \frac{1}{\tg{3t}} * (3 - 3\cos{3t})} =\\
= e^{\lim\limits_{t \to 0}\left(\frac{3 - 3\cos{3t}}{\tg{3t}}\right)} =
e^{3\lim\limits_{t \to 0}\left(\frac{1 - \cos{3t}}{\tg{3t}}\right)} =
\{$используем эквивалентность $1 - \cos{x} \sim \frac{x^2}{2}\} =
e^{3\lim\limits_{t \to 0}\left(\frac{\frac{9t^2}{2}}{\tg{3t}}\right)} =\\
= \{$используем эквивалентность $\tg{x} \sim x \} = 
e^{\frac{3}{2}\lim\limits_{t \to 0}\left(\frac{9t^2}{3t}\right)} =
e^{\frac{3}{2}\lim\limits_{t \to 0} 3t} = e^0 = 1
$
\section{№ 3.}
$\lim\limits_{x \to 0} \frac{(\cos(5\arcsin{2x}) - 1)\log_3{(1 + \sin(\tg^2{4x}))}}{(\sqrt{1 - \arctg^2{6x}} - 1)(5^{\tg{2x^2}} - 1)} = (*)$\\\\
Используем эквивалентности:\\
$\cos(5\arcsin{2x}) - 1 \sim \cos(5 * 2x) - 1 \sim \cos{10x} - 1 \sim -(1 - \cos{10x}) \sim -50x^2$\\
$\sin(\tg^2{4x}) \sim \sin{16x^2} \sim 16x^2$\\
$\arctg^2{6x} \sim 36x^2$\\
$\tg{2x^2} \sim 2x^2$\\
$5^{2x^2} - 1 \sim 2x^2\ln{5}$\\\\
$(*) = \lim\limits_{x \to 0} \frac{-50x^2\log_3{(1 + 16x^2)}}{2x^2\ln{5}\sqrt{1 - 36x^2}} = (*)$\\\\
И снова используем эквивалентности:
$\log_3{(1 + 16x^2)} \sim 16x^2\log_3{e}$\\
$\sqrt{1 - 36x^2} = (1 - 36x^2)^{\frac{1}{2}} \sim -18x^2$\\\\
$(*) = \lim\limits_{x \to 0} \frac{-50x^2 * 16x^2\log_3{e}}{-18x^2 * 2x^2\ln{5}} =  
\frac{\log_3{e}}{\ln{5}}\lim\limits_{x \to 0} \frac{-50 * 8}{-18 * 2} = \frac{\log_3{e}}{\ln{5}} * \frac{50 * 4}{9} = \frac{200\log_3{3}}{9\ln{5}}$\\
\section{№ 4.}
Рассмотрим для каждой точки:\\
1) $a = 1$\\\\
$\lim\limits_{x \to 1-} f(x) = \lim\limits_{x \to 1-} x^2 + 3x = 4$\\
$\lim\limits_{x \to 1+} f(x) = \lim\limits_{x \to 1+} 2x + 2 = 4$\\
$f(1) = 1^2 + 3 * 1 = 4 \Rightarrow f(x)$ непрерывна в точке $1$.\\\\
2) $a = 2$\\\\
$\lim\limits_{x \to 2-} f(x) = \lim\limits_{x \to 2-} 2x + 2 = 6$\\
$\lim\limits_{x \to 2+} f(x) = \lim\limits_{x \to 2+} 2^x = 4 \Rightarrow $ односторонние пределы не равны $\Rightarrow$ имеется разрыв
второго рода.\\
\section{№ 5.}
$f(x) = -5^{\frac{1}{x^2(x^2 + 6x + 8)}}$\\\\
ОДЗ:\\
$x \neq 0$\\
$(x^2 + 6x + 8) \neq 0$\\
$D = 36 - 32 = 4$\\
$x \neq \frac{-6 \textpm 2}{2} = -2; -4$\\\\
$f(x) = -5^{\frac{1}{x^2(x + 4)(x + 2)}}$\\
Пусть $g(x) = \frac{1}{x^2(x + 4)(x + 2)}$, тогда $f(x) = -5^{g(x)}$\\
$f(x)$ - степенная функция. Заметим, что при стремелении аргумента к $-\infty$, значение степенной функции стремится к $0$,
в то время как при стремелении аргумента к $+\infty$, значение степенной функции само стремится к $+\infty$. Этим фактом будем 
пользоваться в дальнейшем.\\
Посчитаем все односторонние пределы:\\
$a = 0$\\
$\lim\limits_{x \to 0-} g(x) = \lim\limits_{x \to 0-} \frac{1}{x^2(x + 4)(x + 2)};\\
x^2 > 0,\\
x+ 4 > 0, \\
x + 2 > 0 \Rightarrow\\
\Rightarrow g(x) > 0$.\\
Вывод: $g(x) \to +\infty \Rightarrow f(x) \to -(+\infty) \Rightarrow f(x) \to -\infty$\\
\\
$\lim\limits_{x \to 0+} g(x) = \lim\limits_{x \to 0+} \frac{1}{x^2(x + 4)(x + 2)};\\
x^2 > 0,\\
x+ 4 > 0, \\
x + 2 > 0 \Rightarrow\\
\Rightarrow g(x) > 0$.\\
Вывод: $g(x) \to +\infty \Rightarrow f(x) \to -\infty$\\
$f(a)$ - неопределено $\Rightarrow a$ - точка разрыва (второго рода).\\
$a = -4$\\
$\lim\limits_{x \to -4 -} g(x) = \lim\limits_{x \to -4 -} \frac{1}{x^2(x + 4)(x + 2)};\\
x^2 > 0,\\
x + 4 < 0, \\
x + 2 < 0 \Rightarrow\\
\Rightarrow g(x) > 0$.\\
Вывод: $g(x) \to +\infty \Rightarrow f(x) \to -(+\infty) \Rightarrow f(x) \to -\infty$\\
\\
$\lim\limits_{x \to -4 +} g(x) = \lim\limits_{x \to -4 +} \frac{1}{x^2(x + 4)(x + 2)};\\
x^2 > 0,\\
x + 4 > 0, \\
x + 2 < 0 \Rightarrow\\
\Rightarrow g(x) < 0$.\\
Вывод: $g(x) \to -\infty \Rightarrow f(x) \to 0$\\
\\
$a = -2$\\
$\lim\limits_{x \to -2 -} g(x) = \lim\limits_{x \to -2 -} \frac{1}{x^2(x + 4)(x + 2)};\\
x^2 > 0,\\
x + 4 > 0, \\
x + 2 < 0 \Rightarrow\\
\Rightarrow g(x) < 0$.\\
Вывод: $g(x) \to -\infty \Rightarrow f(x) \to 0$\\
\\
$\lim\limits_{x \to -2 +} g(x) = \lim\limits_{x \to -2 +} \frac{1}{x^2(x + 4)(x + 2)};\\
x^2 > 0,\\
x + 4 > 0, \\
x + 2 > 0 \Rightarrow\\
\Rightarrow g(x) > 0$.\\
Вывод: $g(x) \to +\infty \Rightarrow f(x) \to -\infty$\\
\section{№ 6.}
$y = e^{3x}\cos{x}$\\
Пусть $f(x) = e^{3x}$, а $g(x) = \cos{x}$, тогда $y = f(x)g(x)$, а\\
$y' = f'(x)g(x) + f(x)g'(x)$\\\\
Найдём $f'(x)$:\\
$f'(x) = \lim\limits_{\Delta{x} \to 0}\frac{f(x + \Delta{x}) - f(x)}{\Delta{x}} = 
\lim\limits_{\Delta{x} \to 0}\frac{e^{3x + 3\Delta{x}} - e^{3x}}{\Delta{x}} =
\lim\limits_{\Delta{x} \to 0}\frac{e^{3x}e^{3\Delta{x}} - e^{3x}}{\Delta{x}} =
\lim\limits_{\Delta{x} \to 0}(e^{3x} * \frac{e^{3\Delta{x}} - 1}{\Delta{x}}) =\\
= \{$используем эквивалентность $e^x - 1 \sim x \} =
\lim\limits_{\Delta{x} \to 0}(e^{3x} * \frac{3\Delta{x}}{\Delta{x}}) = 
3e^{3x}$\\\\
Найдём $g'(x)$:\\
$g'(x) = \lim\limits_{\Delta{x} \to 0}\frac{g(x + \Delta{x}) - g(x)}{\Delta{x}} =
\lim\limits_{\Delta{x} \to 0}\frac{\cos{(x + \Delta{x})} - \cos{x}}{\Delta{x}} =
\lim\limits_{\Delta{x} \to 0}\frac{\cos{x}\cos{\Delta{x}} - \sin{x}\sin{\Delta{x}} - \cos{x}}{\Delta{x}} =\\
= \lim\limits_{\Delta{x} \to 0}\frac{\cos{x}(\cos{\Delta{x}} - 1) - \sin{x}\sin{\Delta{x}}}{\Delta{x}} =
\lim\limits_{\Delta{x} \to 0}\frac{-\cos{x}(1 - \cos{\Delta{x}}) - \sin{x}\sin{\Delta{x}}}{\Delta{x}} =
\lim\limits_{\Delta{x} \to 0}\frac{-\cos{x}\frac{\Delta{x}^2}{2} - \sin{x}\sin{\Delta{x}}}{\Delta{x}} =\\
\{$используем эквивалентность $\sin{x} \sim x\} =
\lim\limits_{\Delta{x} \to 0}\frac{-\cos{x}\frac{\Delta{x}^2}{2} - \Delta{x}\sin{x}}{\Delta{x}} =\\
\lim\limits_{\Delta{x} \to 0}-\cos{x}\frac{\Delta{x}}{2} - \sin{x} = -\sin{x}$\\\\
Наконец найдём $y'$:\\
$y' = f'(x)g(x) + f(x)g'(x) = 3e^{3x}\cos{x} - e^{3x}\sin{x} = e^{3x}(3\cos{x} - \sin{x})$
\section{№ 7.}
$f(x) = \ln\frac{\sin{x}}{\cos{x} + \sqrt{\cos{2x}}} + \sqrt{\arctg{\frac{2}{x}}} + \ctg\sqrt[3]{5}$\\
\\
Разделим эту функцию на несколько:\\
\\
$f_1(x) = \ln\frac{\sin{x}}{\cos{x} + \sqrt{\cos{2x}}}$\\
$f_2(x) = \sqrt{\arctg{\frac{2}{x}}}$\\
$f_3(x) = \ctg\sqrt[3]{5}$\\
\\
таких, что $f(x) = f_1(x) + f_2(x) + f_3(x)$.\\
Соответсвенно: $f'(x) = f'_1(x) + f'_2(x) + f'_3(x)$.\\
Ещё немного расчленим $f_1(x)$:\\
\\
$f_{11}(x) = \frac{\sin{x}}{\cos{x} + \sqrt{\cos{2x}}}$\\
\\
Найдём теперь производную каждой функции:\\
\\
$f'_{11}(x) = \frac{(\sin{x})'(\cos{x} + \sqrt{\cos{2x}}) - (\sin{x})(\cos{x} + \sqrt{\cos{2x}})'}{(\cos{x} + \sqrt{\cos{2x}})^2} =
\frac{\cos{x}(\cos{x} + \sqrt{\cos{2x}}) - \sin{x}(- \sin{x} - \frac{\sin{2x}}{\sqrt{\cos{2x}}})}{(\cos{x} + \sqrt{\cos{2x}})^2} =\\\\
= \frac{\cos^2{x} + \cos{x}\sqrt{\cos{2x}} + \sin^2{x} +  \frac{\sin{x}\sin{2x}}{\sqrt{\cos{2x}}}}{(\cos{x} + \sqrt{\cos{2x}})^2} =
\frac{1 + \frac{\cos{x}\cos{2x} + \sin{x}\sin{2x}}{\sqrt{\cos{2x}}}}{(\cos{x} + \sqrt{\cos{2x}})^2} 
= \frac{1 + \frac{\cos(x - 2x)}{\sqrt{\cos{2x}}}}{(\cos{x} + \sqrt{\cos{2x}})^2} = \frac{\sqrt{\cos{2x}} + \cos{x}}{\sqrt{\cos{2x}}} *
\frac{1}{(\cos{x} + \sqrt{\cos{2x}})^2} =\\
= \frac{1}{(\cos{x} + \sqrt{\cos{2x}})\sqrt{\cos{2x}}}$\\\\
$f'_1(x) = (\ln{f_{11}(x)})' * f'_{11}(x) = \frac{1}{f_{11}(x)} * f'_{11}(x) = 
\frac{\cos{x} + \sqrt{\cos{2x}}}{\sin{x}} * \frac{1}{(\cos{x} + \sqrt{\cos{2x}})\sqrt{\cos{2x}}} = 
\frac{1}{\sin{x}\sqrt{\cos{2x}}}$\\\\
$f'_2(x) = \frac{(\arctg{\frac{2}{x}})'}{2\sqrt{\arctg{\frac{2}{x}}}} = \frac{1}{2\sqrt{\arctg{\frac{2}{x}}}} * 
\frac{1}{1 + \frac{4}{x^2}} * (-\frac{2}{x^2}) = -\frac{1}{(x^2 + 4)\sqrt{\arctg{\frac{2}{x}}}}$\\\\
$f'_3(x) = (\ctg\sqrt[3]{5})' = 0$\\\\
$f'(x) = f'_1(x) + f'_2(x) + f'_3(x) = \frac{1}{(\cos{x} + \sqrt{\cos{2x}})\sqrt{\cos{2x}}} - \frac{1}{(x^2 + 4)\sqrt{\arctg{\frac{2}{x}}}} + 0 =
\frac{(x^2 + 4)\sqrt{\arctg{\frac{2}{x}}} - (\cos{x} + \sqrt{\cos{2x}})\sqrt{\cos{2x}}}
{(\cos{x} + \sqrt{\cos{2x}})(x^2 + 4)\sqrt{\cos{2x}\arctg{\frac{2}{x}}}}$\\
\section{№ 8.}
$y = f(x) = e^x(\cos{2x} + 2\sin{2x})$\\
Найдём $f'(x)$:\\
$f'(x) = (e^x)'(\cos{2x} + 2\sin{2x}) + e^x(\cos{2x} + 2\sin{2x})' = e^x(\cos{2x} + 2\sin{2x} - 2\sin{2x} + 4\cos{2x}) = \\
= 5e^x\cos{2x}$\\
$dy = f'(x) * dx = 5e^x\cos{2x} * dx$\\
Ответ: $dy = 5e^x\cos{2x} * dx$.
\section{№ 9.}
$y = f(x) = \sqrt{2x^2 + 1}$\\\\
$\frac{dy}{dx} = f'(x) \Rightarrow$\\
$\Rightarrow f'(x) =  (\sqrt{2x^2 + 1})' = \frac{(2x^2 + 1)'}{2\sqrt{2x^2 + 1}} = \frac{4x}{2\sqrt{2x^2 + 1}} = \frac{2x}{\sqrt{2x^2 + 1}}
\Rightarrow $\\
$\Rightarrow \frac{dy}{dx} = \frac{2x}{\sqrt{2x^2 + 1}}$\\\\
$\frac{d^2y}{dx^2} = f''(x) \Rightarrow$\\
$\Rightarrow f''(x) = \frac{(2x)`\sqrt{2x^2 + 1} - 2x(\sqrt{2x^2 + 1})'}{2x^2 + 1} = 
\frac{2\sqrt{2x^2 + 1} - 2x * \frac{2x}{\sqrt{2x^2 + 1}}}{2x^2 + 1} = 
\frac{\frac{2(2x^2 + 1) - 4x^2}{\sqrt{2x^2 + 1}}}{2x^2 + 1} = \frac{\frac{4x^2 + 2 - 4x^2}{\sqrt{2x^2 + 1}}}{2x^2 + 1} =
\frac{\frac{2}{\sqrt{2x^2 + 1}}}{2x^2 + 1} =\\\\ = \frac{2}{(2x^2 +1)\sqrt{2x^2 + 1}}$\\\\
Ответ: $\frac{2x}{\sqrt{2x^2 + 1}}$ и $\frac{2}{(2x^2 +1)\sqrt{2x^2 + 1}}$.
\section{№ 10.}
$f'(x) = \frac{1}{(x + 1)\ln{10}}$\\
$f''(x) = -\frac{1}{(x + 1)^2\ln{10}}$\\
$f'''(x) = \frac{2}{(x + 1)^3\ln{10}}$\\
$f''''(x) = -\frac{6}{(x + 1)^4\ln{10}}$\\
$f'''''(x) = \frac{24}{(x + 1)^5\ln{10}}$\\
\textbf{Предположение:} $\frac{d^ny}{dx^n} = (-1)^{n + 1}\frac{(n - 1)!}{(x + 1)^n\ln{10}}$.\\
Докажем предположение по индукции:\\
База: $n = 1$\\
$f^{(1)}(x) = (\lg(1 + x))' = \frac{1}{(x + 1)\ln{10}}$\\
$(-1)^{n + 1} * \frac{(n - 1)!}{(x + 1)^n\ln{10}} = \frac{0!}{(x + 1)\ln{10}} = \frac{1}{(x + 1)\ln{10}}$\\
$\frac{1}{(x + 1)\ln{10}} = \frac{1}{(x + 1)\ln{10}}$ $ \blacksquare$\\
\\
Предположение индукции: $n = k$:\\
$f^{(k)}(x) = (-1)^{k + 1} * \frac{(k - 1)!}{(x + 1)^k\ln{10}}$.\\
\\
Шаг индукции: $n = k +1$\\
$f^{(k + 1)}(x) = (-1)^{k + 2} * \frac{k!}{(x + 1)^{k + 1}\ln{10}}$.\\
$(f^{(k)}(x))' = (-1) * (-1)^{k + 1} * \frac{k(k - 1)!}{(x + 1)^k(x + 1)\ln{10}}$\\
$(f^{(k)}(x))' = -\frac{k}{(x + 1)} * (-1)^{k + 1} * \frac{(k - 1)!}{(x + 1)^k\ln{10}}$\\
$(f^{(k)}(x))' = -\frac{k}{(x + 1)} * f^{(k)}(x)$\\
\end{document}
