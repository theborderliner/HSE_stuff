\documentclass[a4paper, 12pt]{report}

\usepackage[T2A]{fontenc}
\usepackage[utf8]{inputenc}
\usepackage[english,russian]{babel}
\usepackage{amsmath} %математические пакеты
\usepackage{amsfonts}
\usepackage{amssymb}
\usepackage[pdftex,unicode]{hyperref}
\usepackage[dvipsnames]{xcolor}
\usepackage{graphicx}

\usepackage[top=2cm,
left=2cm,
right=2cm,
bottom=2cm]{geometry}

\title{Большое домашнее задание № 3. Математический анализ.}
\author{Пешехонов Иван. БПМИ1912}
\date{\today}

\begin{document}
\maketitle
\clearpage
\chapter{Вариант 19.}
\section{№ 1.}
Найти предел по правилу Лопиталя:\\
\\
a) $\lim\limits_{x\to{0}} \frac{(x + 1)\sin^2{x}}{(x + a)\ln^2{(x + 1)}} = 
\lim\limits_{x\to{0}} \frac{((x + 1)\sin^2{x})'}{((x + a)\ln^2{(x + 1)})'} =
\lim\limits_{x\to{0}} \frac{\sin^2{x} + 2(x + 1)\sin{x}\cos{x}}{\ln^2(x + 1) + 2(x + a)\frac{\ln(x + 1)}{x + 1}} =
\lim\limits_{x\to{0}} \frac{\sin^2{x} + (x + 1)\sin{2x}}{\frac{(x+1)\ln^2(x + 1) + 2(x + a)\ln(x + 1)}{x + 1}} =\\
=\lim\limits_{x\to{0}} \frac{(x + 1)(\sin^2{x} + (x + 1)\sin{2x})}{(x+1)\ln^2(x + 1) + 2(x + a)\ln(x + 1)} =
\lim\limits_{x\to{0}} \frac{((x + 1)(\sin^2{x} + (x + 1)\sin{2x}))'}{((x+1)\ln^2(x + 1) + 2(x + a)\ln(x + 1))'} =\\
= \lim\limits_{x\to{0}} \frac{\sin^2{x} + (x + 1)\sin{2x} + (x + 1)(\sin{2x} + \sin{2x} + 2(x + 1)\cos{2x})}
{\ln^2(x + 1) + 2(x + 1)\frac{\ln{(x + 1)}}{x + 1} + 2(\ln{(x + 1)} + \frac{(x + a)}{x + 1})} =
\lim\limits_{x\to{0}} \frac{\sin^2{x} + (x + 1)\sin{2x} + 2(x + 1)(\sin{2x}+ (x + 1)\cos{2x})}
{\ln^2(x + 1) + 2\ln{(x + 1)} + 2(\ln{(x + 1)} + \frac{(x + a)}{x + 1})} =\\
= \lim\limits_{x\to{0}} \frac{0 + 0 + 2(0 + 1)}
{0 + 0 + 2(0 + a)} = \frac{1}{a}$\\\\
б) $\lim\limits_{x \to 0}\left(\frac{1}{1 - \cos{x}} - \frac{1}{e^x - e^{-x}}\right) = 
\lim\limits_{x \to 0}\left(\frac{1}{1 - \cos{x}}\right) - \lim\limits_{x \to 0}\left(\frac{1}{e^x - e^{-x}}\right) = 
\lim\limits_{x \to 0}\left(\frac{(1)'}{(1 - \cos{x})'}\right) - \lim\limits_{x \to 0}\left(\frac{(1)'}{(e^x - e^{-x})'}\right) =\\\\
= \lim\limits_{x \to 0}\left(\frac{0}{\sin{x}}\right) - \lim\limits_{x \to 0}\left(\frac{0}{e^x + e^{-x}}\right) =
\lim\limits_{x \to 0}\left(\frac{(0)'}{(\sin{x})'}\right) - \lim\limits_{x \to 0}\left(\frac{0}{1 + 1}\right) =
\lim\limits_{x \to 0}\left(\frac{0}{\cos{x}}\right) - \lim\limits_{x \to 0}0 =
\lim\limits_{x \to 0}0 - \lim\limits_{x \to 0}0 =\\\\
= 0 - 0 = 0
$\\\\
в) $\lim\limits_{x \to +\infty}\left(\frac{\ln{x}}{x}\right)^\frac{\cos{x}}{x} = 
\lim\limits_{x \to +\infty}e^{\ln{\left(\frac{\ln{x}}{x}\right)^\frac{\cos{x}}{x}}} =
e^{\lim\limits_{x \to +\infty}\frac{\cos{x}}{x}\ln{\left(\frac{\ln{x}}{x}\right)}}= 
e^{\lim\limits_{x \to +\infty}\frac{\cos{x}}{x}\ \cdot \lim\limits_{x \to +\infty}\ln{\left(\frac{\ln{x}}{x}\right)}} =\\
= e^{\lim\limits_{x \to +\infty}\frac{\cos{x}}{x}\ \cdot \ln\lim\limits_{x \to +\infty}{\left(\frac{\ln{x}}{x}\right)}} =
e^{\lim\limits_{x \to +\infty}\frac{\cos{x}}{x}\ \cdot \ln\lim\limits_{x \to +\infty}{\left(\frac{(\ln{x})'}{(x)'}\right)}} =
e^{\lim\limits_{x \to +\infty}\frac{\cos{x}}{x}\ \cdot \ln\lim\limits_{x \to +\infty}{\left(\frac{1}{x}\right)}} =
e^{\lim\limits_{x \to +\infty}\frac{\cos{x}}{x}\ \cdot \ln{0}} =\\
= e^{\lim\limits_{x \to +\infty}\frac{\cos{x}}{x}\ \cdot (-\infty)}$\\
И вот как бы дальше не понятно. Кажестся, что $\cos{x}$ - ограничена константой, и предел будет $\infty$,
и соответсвенно буде $\infty \cdot (-\infty)$, что неопределённостью не является.\\
С другой стороны при некоторых $x$,\ $\cos{x} = 0 $, и тогда получается $0 \cdot (-\infty)$, что
уже неопределённость.\\
В общем либо ты понял что я тут написал, и это чего-то стоит, либо давай на защиту...\\
\section{№ 2.}
$f(x) = (1 - x^2)e^{x-1}, x_0 = 1;$\\
\\
$f'(x) = (1 - x^2)'e^{x - 1} + (1 - x^2)[e^{x - 1}]' = -2xe^{x - 1} + (1 - x^2)e^{x - 1} = (1 - 2x - x^2)e^{x - 1}$\\
$f''(x) = [1 - 2x - x^2]'e^{x - 1} + (1 - 2x - x^2)[e^{x - 1}]' = (-2 - 2x)e^{x - 1} + (1 - 2x - x^2)e^{x - 1} =\\
= (-1 - 4x - x^2)e^{x - 1}$\\
$f'''(x) = [-1 - 4x - x^2]'e^{x - 1} + (-1 - 4x - x^2)[e^{x - 1}]' = (-4 - 2x)e^{x - 1} + (-1 - 4x - x^2)e^{x - 1} =\\
= (-5 - 6x - x^2)e^{x - 1}$\\
$f(x_0) =  0$\\
\\
Формула Тейлора:\\
$f(x) = f(x_0) + \sum\limits_{k = 1}^n{\frac{f^{(k)}(x_0)}{k!}(x - x_0)^k} + \overline{\overline{o}}((x - x_0)^3)$\\
\\
$f(x) = 0 + f'(x_0)(x - x_0) + \frac{f''(x_0)}{2}(x - x_0)^2 + \frac{f'''(x_0)}{6}(x - x_0)^3 + \overline{\overline{o}}((x - x_0)^3) =\\
= -2(x - x_0) + \frac{-6}{2}(x - x_0)^2 + \frac{-12}{6}(x - x_0)^3 + \overline{\overline{o}}((x - x_0)^3) = \\
= -2(x - x_0) - 3(x - x_0)^2 - 2(x - x_0)^3 + \overline{\overline{o}}((x - x_0)^3)$
\section{№ 3.}
$f(x) = x\arctg{x} - \ln{\sqrt{1 - x^2}}, x_0 = 0;$\\
\\
$g(x) = \ln{\sqrt{1 - x^2}} = \ln{[1 + (\sqrt{1 - x^2} - 1)]}$\\
$h(x) = \arctg{x}$\\
$f(x) = xh(x) - g(x)$\\
$\varsigma = (\sqrt{1 - x^2} - 1) \Rightarrow g(x) = \ln(1 + \varsigma)$\\
$g(x) = \varsigma - \frac{\varsigma^2}{2} + \frac{\varsigma^3}{3} + ... + (-1)^{n - 1}\frac{\varsigma^n}{n}$\\
$h(x) = x - \frac{x^3}{3} + \frac{x^5}{5} - ... + (-1)^{n - 1}\frac{x^{2n - 1}}{2n - 1}$\\
$xh(x) = x^2 - \frac{x^4}{3} + \frac{x^6}{5} - ... + (-1)^{n - 1}\frac{x^{2n}}{2n - 1}$\\
$f(x) = \sum\limits_{k = 1}^n[(-1)^{k - 1}\frac{\varsigma^k}{k} + (-1)^k\frac{x^{2k}}{2k - 1}]$
\section{№ 4.}
$\lim\limits_{x \to 0}\frac{x^2e^{x^2} - \sqrt{x - 5x^2}}{\tg(x^3)} = \lim\limits_{x \to 0}\frac{x^2e^{x^2} - \sqrt{x(1 - 5x)}}{\tg(x^3)} = 
\lim\limits_{x \to 0}\frac{x^2e^{x^2} - \sqrt{x}\sqrt{1 - 5x}}{\tg(x^3)}$\\
\\
$f(x) = e^{x^2}$\\
$g(x) = \sqrt{1 - 5x}$\\
$h(x) = \tg(x^3)$\\
\\
$f(x) = 1 + x^2 + \overline{\overline{o}}(x^3)$\\
$g(x) = 1 - \frac{5x}{2} + \overline{\overline{o}}(x)$\\
$h(x) = x^3 + \frac{x^9}{3} + \overline{\overline{o}}(x^{26})$\\
\\
Найдём формулу Маклорена для числителя:\\
$x^2e^{x^2} - \sqrt{x}\sqrt{1 - 5x} = x^2(1 + x^2) - \sqrt{x}(1 - \frac{5x}{2}) = x^2 + x^4 - \sqrt{x} + \frac{5x\sqrt{x}}{2} =\\
= -\sqrt{x}(1 - x\sqrt{x} - x^3\sqrt{x} - \frac{5x}{2})$
\\
\\
Подставим в дробь:\\
$\lim\limits_{x \to 0}\frac{-\sqrt{x}(1 - x\sqrt{x} - x^3\sqrt{x} - \frac{5x}{2})}{x^3 + \frac{x^9}{3}} = 
\lim\limits_{x \to 0}-\frac{\sqrt{x}(1 - x\sqrt{x} - x^3\sqrt{x} - \frac{5x}{2})}{\sqrt{x}({x^2\sqrt{x} + \frac{x^8\sqrt{x}}{3})}} = 
\lim\limits_{x \to 0}-\frac{(1 - x\sqrt{x} - x^3\sqrt{x} - \frac{5x}{2})}{{x^2\sqrt{x} + \frac{x^8\sqrt{x}}{3}}} = 
\lim\limits_{x \to 0}-\infty = -\infty$\\
\\
Обоснование:\\
1) $(1 - x\sqrt{x} - x^3\sqrt{x} - \frac{5x}{2}) \to 1$, при $x \to 0$ (очевидно).\\
2) Знаменатель $\to$ 0.\\
\section{№ 5.}
Исследовать ф-цию с помощью производной первого порядка.\\
\\
$y = \frac{1}{e^x - 1}$\\
\textbf{ОДЗ}: $(-\infty; 0)\cup(0; +\infty)$\\
\textbf{Пересечение с нулями}:\\
$\frac{1}{e^x - 1} = 0$\\
$1 = 0\ \Rightarrow $ пересечение с нулями \textbf{отсутвует}.\\
\textbf{Промежутки знакопостоянтсва}:\\
При $x < 0 \Rightarrow y < 0$;\\
При $x > 0 \Rightarrow y > 0$;\\
\textbf{Производная}:\\
$y' = ((e^x - 1)^{-1})' = -e^x(e^x - 1)^{-2} = -\frac{e^x}{(e^x - 1)^2}$\\
\textbf{Точки экстремума}:\\
$-\frac{e^x}{(e^x - 1)^2} = 0$\\
$-e^x = 0\ \Rightarrow $ экстремумы \textbf{отсутствуют}.
\[
\begin{tabular}{| c | c |}
x < 0 & x > 0\\
y' < 0 & y' < 0\\
y(x) - убывает & y(x) - убывает
\end{tabular}
\]
\includegraphics[scale=0.3]{plot3.png}
\end{document}
