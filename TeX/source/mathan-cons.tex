\documentclass[a4paper,11pt]{report}

\usepackage[T2A]{fontenc}
\usepackage[utf8]{inputenc}
\usepackage[english,russian]{babel}
\usepackage{amsmath} %математические пакеты
\usepackage{amsfonts}
\usepackage{amssymb}
\usepackage[pdftex,unicode]{hyperref}
\usepackage[dvipsnames]{xcolor}

\usepackage[top=2cm,
left=2cm,
right=2cm,
bottom=2cm]{geometry}
%\renewcommand{\familydefault}{\sfdefault}

\title{Матан. Консультация}
\author{Пешехонов Иван. БПМИ1912}
\date{\today}

\begin{document}
\chapter{Консультация по матану. 22.11.2019}
\section{Немного теории и указания к БДЗ 2.}
$\lim\limits_{x\to0} (1+x)^{1/x} = e$\\
$\lim\limits _{x\to0}f(x)^{g(x)}$ представить в виде $ f(x) = 1 + h(x)$, где $h(x)\to0$\\
\\
\\
$f'(x) = \lim\limits_{x\to0}{\frac{f(x + \Delta{x}) - f(x)}{\Delta{x}}}$\\
$f(x) = x^n$\\
$\lim\limits_{\Delta{x}\to0}\frac{(x + \Delta{x})^n - x^n}{\Delta{x}} = \lim\limits_{\Delta{x}\to0}\frac{\Delta{x}(...)}{\Delta{x}}$ и теперь $\Delta{x}$ можно сократить.\\
\\
\\
$dy = f'(x)*dx$ - формула дифференциала.\\
$\frac{dy}{dx} = f'(x)$ - вывод производной через дифференциал.\\
\\
\\
$f(x)$ - непрерывна в точке $a$, если $\lim\limits_{x\to{a}}f(x) = f(a)$. В данном случае имеется ввиду,
что у функции \underline{существуют} левый и правый пределы, кроме того
они \underline{равны}:\\
\\
$\lim\limits_{x\to{a-0}}{f(x)} = \lim\limits_{x\to{a+0}}{f(x)}=f(a)$\\
\\
\\
В домашке, в 4 номере важно проверить существование обоих пределов, а то пизда.\\
\\

Разрыв первого типа: когда пределы справа и слева существуют, но они не равны. 
Пример: ну хотя бы отображение, которое переводит отрицательные числа в $-1$, 
положительные в $1$, а $0$ в $0$.\\

Разрыв второго типа: ходя бы один из односторонних пределов уходит в бесконечность.
Пример: $\frac{1}{x}$\\

Разрыв третьего типа (устранимый разрыв): самый привычный нам разрыв. 
Пример: $f(x) = x$, то функция, например, не определена в точке $x = 1$. 
Тогда эта точка на графике будет выколота, и это как раз и будет являться устранимым разрывом.\\
\\
\textbf{Важно:} в домашке называть устранимый разрыв именно \underline{устранимым разрывом},
а не разрывом третьего типа. \textbf{Тоже важно:} при построении эскиза графика явно обозначать
устранимый разрыв так, чтобы он был заметен, а не выглядел мелкой точкой.\\
\\
\section{Эквивалентности:}
$
\sin{x} \sim x, x\to0\\
\cos{x} \sim \frac{1 - x^2}{2}, x\to0 \\
\ln(1+x) \sim 1, x\to0 \\
e^x \sim 1 + x, x\to0 \\
(1 + x)^a \sim 1 + ax, x\to0 \\
$\\
Когда хочется использовать в решении эквивалентность, следует явно писать:
"\textcolor{red}{используем эквивалентность: $\sin{x}=x, x\to0$}"\\
\\
\section{Неопределённости:}
$\frac{0}{0}, \frac{\infty}{\infty}, \infty - \infty, 1^{\infty}, 0^{\infty}$\\
$\infty + \infty = \infty$ - это не неопределённость!!!\\
\end{document}
