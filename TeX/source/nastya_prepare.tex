\documentclass[a4paper,11pt]{report}
\usepackage[T2A]{fontenc}
\usepackage[utf8]{inputenc}
\usepackage[english,russian]{babel}
\usepackage{amsmath} %математические пакеты
\usepackage{amsfonts}
\usepackage{amssymb}
\usepackage[pdftex,unicode]{hyperref}
\usepackage[top=2cm,
left=2cm,
right=2cm,
bottom=2cm]{geometry}
\usepackage[dvipsnames]{xcolor}

\DeclareMathOperator{\real}{\mathbb{R}}
\DeclareMathOperator{\Mnm}{\real^{n\times m}}
\DeclareMathOperator{\Mmn}{\real^{m\times n}}
\DeclareMathOperator{\Mn}{\real^{n\times n}}

\title{Дискра ДЗ.}
\author{Пешехонов Иван. БПМИ1912}
\date{\today}

\begin{document}
\chapter{Задачи для подготовки к экзамену. Ленал.}
\section{№ 1.1}
Найти все матрицы $X \in \mathbb{R}^{3\times3}$, удовлетворяющие уравнению $AX = B$, где
\[
A = 
\begin{pmatrix}
1 & 2 & 1\\
2 & 3 & -1\\
1 & 1 & 2\\
\end{pmatrix},
B = 
\begin{pmatrix}
4 & 3 & 6\\
-2 & 8 & 7\\
6 & 1 & 5\\
\end{pmatrix}
\]
\textbf{Пример:}\\
Рассмотрим матричное уравнение $AX = B$, где
\[
A = 
\begin{pmatrix}
1 & -1 & -2\\
1 & -2 & -2\\
2 & -3 & -4\\
\end{pmatrix},
B = 
\begin{pmatrix}
0 & 1\\
-1 & -1\\
-1 & 0\\
\end{pmatrix}
\].
Записываем расширенную матрицу и приводим её к улучшенному ступенчатому виду:
\[
\begin{pmatrix}
\begin{tabular}{1 1 1|1 1}
1 & -1 & -2 & 0 & 1\\
1 & -2 & -2 & -1 & -1\\
2 & -3 & -4 & -1 & 0\\
\end{tabular}
\end{pmatrix}
\longrightarrow
\begin{pmatrix}
\begin{tabular}{1 1 1|1 1}
1 & 0 & -2 & 1 & 3\\
0 & -1 & 0 & -1 & -2\\
0 & 0 & 0 & 0 & 0\\
\end{tabular}
\end{pmatrix}
\]
Неприятность заключается в том, что слева у нас получилась не единичная матрица, а какая-то прямоугольная.
Пусть искомая матрица $X$ равна 
$
\begin{pmatrix}
x_1 & x_4\\
x_2 & x_5\\
x_3 & x_6\\
\end{pmatrix}
$. К сожалению мы не умеем решать такие СЛУ, где слева и справа одновременно стоят неквадратные матрицы,
зато умеем решать СЛУ, где справа стоит одинокий вектор-столбец. Этим и займёмся: разобьём нашу расширенную матрицу
на две
\[
\begin{pmatrix}
\begin{tabular}{1 1 1|1}
1 & 0 & -2 & 1\\
0 & -1 & 0 & -1\\
\end{tabular}
\end{pmatrix}
\\
\begin{pmatrix}
\begin{tabular}{1 1 1|1}
1 & 0 & -2 & 3\\
0 & -1 & 0 & -2\\
\end{tabular}
\end{pmatrix}
\]
и будем решать их отдельно. В итоге решение первой матрицы даст нам выражение первого столбца матрицы $X$,
а решение второй, соответсвенно, второго.\\
Можешь поверить мне на слово, или проверить, но итоговым решением будет:\\
\[X = 
\begin{pmatrix}
1 + 2x_3 & 3 + 2x_6\\
1 & 2\\
x_3 & x_6\\
\end{pmatrix}
, x_3, x_6 \in \real
\]\\
\textbf{Важно} не забыть, когда решаешь системы отдельно, о том, что свободные переменные у них, вообще говоря,
разные, и если ты их обозначаешь зачем-то другой буквой (какая-нибудь $\beta$ вместо $x_1$), 
то каждую переменную надо обозначать разными буквами.\\
\section{№ 1.2}
Найти все матрицы $X \in \mathbb{R}^{3\times3}$, удовлетворяющие уравнению $AX = B$, где
\[
A = 
\begin{pmatrix}
3 & 1 & 0\\
2 & 4 & 5\\
0 & 2 & 3\\
\end{pmatrix},
B = 
\begin{pmatrix}
-1 & 1 & -4\\
1 & 9 & -1\\
1 & 5 & 1\\
\end{pmatrix}
\]
\section{№ 1.3}
Постарайся вспомнить, как решать такое уравнение:\\
$XA = B$, где
\[
A = 
\begin{pmatrix}
1 & -1 & -2\\
-2 & -3 & -2\\
1 & -1 & 2\\
\end{pmatrix},
B = 
\begin{pmatrix}
4 & -6 & 0\\
5 & -6 & 0\\
\end{pmatrix}
\]
\section{№ 1.4 (скорее всего не будет на экзамене, но вдруг...)}
Решить матричное уравнение:
\[
\begin{pmatrix}
1 & -1 & -2\\
1 & 3 & -2\\
1 & -1 & 2\\
\end{pmatrix}
X
\begin{pmatrix}
-1 & -1\\
1 & 2\\
\end{pmatrix}
=
\begin{pmatrix}
5 & 2\\
3 & -4\\
-3 & -2\\
\end{pmatrix}
\]
\textbf{Подсказка:} если будет грустно, попробуй что-нибудь на что-нибудь заменить. Если будет очень грустно,
напиши мне :D
\section{№ 1.5 (тоже наверняка не будет на экзамене)}
\[
\begin{pmatrix}
1 & 1\\
1 & 1\\
\end{pmatrix}
X
\begin{pmatrix}
1 & 0\\
0 & 0\\
\end{pmatrix}
=
\begin{pmatrix}
3 & 0\\
3 & 0\\
\end{pmatrix}
\]
\section{№ 2.1}
Найти все комплексные решения уравнения $(\sqrt{3} - 2i)z^3 = -\sqrt{2} + 3\sqrt{6}i$.\\
\section{№ 2.2}
Найти все комплексные решения уравнения $(2 + \sqrt{3}i)z^3 = 3\sqrt{6} + \sqrt{2}i$.\\
\section{№ 2.3}
Найти все комплексные решения уравнения $z^6 - (\sqrt{2} - 2\sqrt{6}i)z^3 - 6 - 2\sqrt{3}i = 0$.\\
\textbf{Подсказка:} вспомни, как решала биквадратные уравнения в школе.
\section{№ 2.4}
Найти все комплексные решения уравнения $z^6 + (\sqrt{6} - 2\sqrt{2}i)z^3 - 2 - 2\sqrt{3}i = 0$.\\
\section{№ 3.1}
Выяснить, будут ли векторы\\
$a_1 = (1, 1, -1, 2)$\\
$a_2 = (2, 0, 1, -1)$\\
$a_3 = (1, -1, -2, 3)$\\
линейно независимы в пространстве $R^4$.
\section{№ 3.2}
Аналогично про векторы\\
$f_1 = -x^2 + 2x + 3$\\
$f_2 = x^2 - x + 2$\\
$f_3 = 2x^2 + 3x + 1$\\
в пространстве $R[x]_2$.
\section{№ 3.3} 
Аналогично про векторы\\
$A_1 =
\begin{pmatrix}
1 & -1\\
1 & 2\\
\end{pmatrix}
A_2 =
\begin{pmatrix}
-1 & 1\\
1 & 1\\
\end{pmatrix}
A_3 =
\begin{pmatrix}
3 & -3\\
1 & 3\\
\end{pmatrix}
$\\
в пространстве $\mathbb{R}^{2\times2}$
\section{№ 3.4}
Выяснить, будет ли вектор $b = (3, 1, 4)$ принадлежать 
линейной оболочке векторов\\
$a_1 = (1, 1, 1)$
$a_2 = (1, -1, 2)$
$a_3 = (1, -3, 3)$
\section{№ 3.5}
Доказать, что функции $\cos{x}, \cos{2x}, \cos{3x}$ линейно независимы.
\section{№ 3.6}
Выяснить, принадлежит ли функция $\sin{3x}$ линейной оболочке функций $\sin{x}, \cos{x}, \cos^3{x}$.
\section{№ 3.7}
Выяснить, принадлежит ли функция $\cos{3x}$ линейной оболочке функций $\sin{x}, \cos{x}, \sin^3{x}$.\\
\\
\textbf{Пример:}\\
Пусть даны столбцы матрицы $A \in \mathbb{R}^{4\times4}$ равные $a_1, a_2, a_3, a_4$.\\
Положим:
\[
b_1 = a_1 + 2a_2,\ 
b_2 = a_1 + 3a_2 + a_3,\ 
b_3 = -a_1 + a_2 + a_3 + 2a_4,\ 
b_4 = a_1 + 2a_2 + 3a_3 - a_4.\ 
\]
Чему равен определитель матрицы $B$ со столбцами $b_1, b_2, b_3, b_4$, если определитель матрицы $A$ равен 5?\\
\textbf{Решение:}
На самом деле это очень простое задание на элементарные преобразования, в котором просто есть парочка хитростей.\\
Прежде всего нам дан определитель матрицы $A$, запишем его:
\[
|A| = |a_1\ a_2\ a_3\ a_4| = 5
\]
Теперь запишем определитель матрицы $B$:
\[
|B| = |b_1\ b_2\ b_3\ b_4| = \Arrowvert[a_1 + 2a_2]\ [a_1 + 3a_2 + a_3]\ [-a_1 + a_2 + a_3 + 2a_4]\ [a_1 + 2a_2 + 3a_3 - a_4]\Arrowvert
\]
Я заключил каждый столбец в квадратные скобки, потому что иначе нихера не понятно, где заканчивается один столбец, и начинается
другой. Теперь, когда мы это записали, начинаем выполнять элементарные преобразования столбцов. Я не умею,
как Федотов, красиво обозначать тип преобразований, поэтому буду как-нибудь буковками.\\
Прежде всего вычтем из второго столбца первый, а к третьему добавим четвёртый:
\\\\
$
\Arrowvert[a_1 + 2a_2]\ [a_1 + 3a_2 + a_3]\ [-a_1 + a_2 + a_3 + 2a_4]\ [a_1 + 2a_2 + 3a_3 - a_4]\Arrowvert =\\
\Arrowvert[a_1 + 2a_2]\ [a_2 + a_3]\ [3a_2 + 4a_3 + a_4]\ [a_1 + 2a_2 + 3a_3 - a_4]\Arrowvert
$
\\\\
Теперь из третьего столбца вычтем 3 вторых, а из последнего вычтем первый:
\[
\Arrowvert[a_1 + 2a_2]\ [a_2 + a_3]\ [3a_2 + 4a_3 + a_4]\ [a_1 + 2a_2 + 3a_3 - a_4]\Arrowvert =
\Arrowvert[a_1 + 2a_2]\ [a_2 + a_3]\ [a_3 + a_4]\ [3a_3 - a_4]\Arrowvert
\]
Прибавляем к последнему столбцу третий:
\[
\Arrowvert[a_1 + 2a_2]\ [a_2 + a_3]\ [a_3 + a_4]\ [3a_3 - a_4]\Arrowvert =
\Arrowvert[a_1 + 2a_2]\ [a_2 + a_3]\ [a_3 + a_4]\ [4a_3]\Arrowvert
\]
И теперь воспользуемся свойством определителя, по которому из каждого элемента определённого столбца можно вынести
одно и то же число за знак определителя (\textbf{важно не продолбать число!}):
\[
\Arrowvert[a_1 + 2a_2]\ [a_2 + a_3]\ [a_3 + a_4]\ [4a_3]\Arrowvert =
4\Arrowvert[a_1 + 2a_2]\ [a_2 + a_3]\ [a_3 + a_4]\ [a_3]\Arrowvert
\]
Теперь всё стало совсем просто, вычитаем из второго и третьего столбца последний:
\[
4\Arrowvert[a_1 + 2a_2]\ [a_2 + a_3]\ [a_3 + a_4]\ [a_3]\Arrowvert =
4\Arrowvert[a_1 + 2a_2]\ [a_2]\ [a_4]\ [a_3]\Arrowvert
\]
И теперь вычитаем из первого два вторых:
\[
4\Arrowvert[a_1 + 2a_2]\ [a_2]\ [a_4]\ [a_3]\Arrowvert =
4|a_1\ a_2\ a_4\ a_3|
\]
Полученный определитель подозрительно похож на определитель матрицы $A$, разве что последние два столбца стоят не на своих
местах. Хорошо, что у нас есть элементарное преобразование второго типа, которое позволяет нам поменять местами два столбца
определителя \textbf{при этом не забыв поменять его знак}:
\[
4|a_1\ a_2\ a_4\ a_3| = 
-4|a_1\ a_2\ a_3\ a_4| =
-4 \cdot 5 = -20
\]
Ответ: $-20$.
\section{№ 3.8}
Даны столбцы матрицы $A \in \mathbb{R}^{4\times4}$, равные $a_1, a_2, a_3, a_4$.\\
Положим:
\[
b_1 = -a_1 + 3a_2,\ 
b_2 = a_1 - 4a_2 - a_3,\ 
b_3 = a_1 - a_2 + 2a_3 - a_4,\ 
b_4 = a_1 + a_2 - 3a_3 + 2a_4
\]
Чему равен определитель матрицы $B$ со столбцами $b_1, b_2, b_3, b_4$, если определитель матрицы
$A$ равен 4?
\section{№ 3.9 (*)} 
Выяснить, будут ли векторы\\
$f_1(x) = 1$\\
$f_2(x) = 2^x$\\
$f_3(x) = 3^x$\\
$\vdots$\\
$f_n(x) = n^x$\\
линейно независимы в пространстве $C[0, 1]$ (непрерывных функций на отрезке $[0, 1]$).
\section{№ 4.1}
Доказать, что множество всех матриц $X \in \mathbb{R}^{2\times2}$ удовлетворяющих условию $tr(XY) = 0$, где
$Y = 
\begin{pmatrix}
2 & 3\\
4 & 5\\
\end{pmatrix}
$, является подпространством в пространстве $\mathbb{R}^{2\times2}$; найти базис и размерность этого подпространства.\\
\textbf{Подсказка № 1:} обрати внимание, что сначала тебе нужно именно доказать, и не надо для этого искать все такие матрицы 
$X$ в общем виде. Подумай, там правда несложно доказывается.\\
\textbf{Подсказка № 2:} след матрицы равен нулю не только когда элементы на главной диагонали равны нулю.
\section{№ 4.2}
Доказать, что множество всех матриц $X \in \mathbb{R}^{2\times2}$ удовлетворяющих условию $tr(YX) = 0$, где
$Y = 
\begin{pmatrix}
4 & 6\\
8 & 10\\
\end{pmatrix}
$, является подпространством в пространстве $\mathbb{R}^{2\times2}$; найти базис и размерность этого подпространства.\\
\section{№ 4.3} 
Пусть V - векторное пространство всех многочленов степени не выше 4 с действительными коэффициентами, и пусть
$U \subseteq V$ - подмножество, состоящее из всех многочленов $f(x)$, удовлетворяющих условиям:\\
1) $2f(-1) = 3f'(1)$\\
2) $f''(\frac{1}{2}) = 0$\\
Доказать, что $U$ - подпространство в $V$.\\
Найти базис и размерность этого подпространства.\\
\\
\textbf{Пояснение:} не совсем очевидное доказательство, поэтому пример я сейчас приведу:\\
Во-первых, запишем ``многочлен степени не выше 4" в явном виде: $f(x) = a_4x^4 + a_3x^3 + a_2x^2 + a_1x + a_0$.\\
Ну, как мы проверяем, что что-то - подпространство? Ну, по определению, а именно проверяем три свойства:\\
1) Принадлежость нулевого вектора подпространству. В данном случае это очевидно так, т.к. полагая $a_i = 0$ мы и получаем
этот самый нулевой вектор.\\
2) Сумма векторов снова должна принадлежать подпространству. Проверяем: \\
\[
\begin{cases}
 2f(-1) = 3f'(1)\\
 2g(-1) = 3g'(1)\\
\end{cases}
\Rightarrow
2f(-1) + 2g(-1) = 3f'(1) + 3g'(1)
\Rightarrow
2(f + g)(-1) = 3(f' + g')(1)
\Rightarrow 
f + g \in U
\]
3) Произведение вектора на скаляр снова должно принадлежать подпространству. Проверяем:\\
\[
\lambda(2f(1)) = \lambda(3f'(1)) 
\Rightarrow
2(\lambda{f}(1)) = 3(\lambda{f'}(1)) 
\Rightarrow 
\lambda{f} \in U
\]
Всё, проверили все условия, всё выполняется, значит $U$ - действительно подпространство, можешь искать базис)
\section{№ 4.4}
Пусть V - векторное пространство всех многочленов степени не выше 4 с действительными коэффициентами, и пусть
$U \subseteq V$ - подмножество, состоящее из всех многочленов $f(x)$, удовлетворяющих условиям:\\
1) $2f(1) = f'(-1)$\\
2) $f''(-\frac{1}{2}) = 0$\\
Доказать, что $U$ - подпространство в $V$.\\
Найти базис и размерность этого подпространства.
\\
\\
\textbf{Что такое хорошо, а что такое плохо.}\\
Разберёмся сейчас, в каких случаях надо записывать векторы в матрицу как столбцы, а когда как строки, а так же
когда делать преобразования столбцов, а когда строк.\\
На самом деле смысл именно в преобразованиях, потому что векторы в матрицу записать можно как угодно, но
от выбранного тобой способа записи будет зависить тип преобразований, которые будет правильно использовать.\\
Разберём задачи по типу:\\
1) \textbf{Дана} система векторы, надо \textbf{найти какой-нибудь} базис подпространства, натянутого на их
линейную оболочку. В этом случае ты будешь работать с конкретной линейной оболочкой, и тебе важно не 
вылезти за её границу. Очевидный вопрос: а в каких случаях можно вылезти за линейную оболочку? 
Очевидный ответ: в случаях, когда ты прибавляешь к векторам из линейной оболочки векторы,
которые в ней не лежат. Например даны тебе два вектора, которые ты решила записать в строки матрицы:
\[
\begin{pmatrix}
1 & 2 & 3 & 4\\
2 & 1 & 3 & 7\\
\end{pmatrix}
\]
И дальше начинаешь делать элементарные преобразования столбцов (например прибавляешь к первому слобцу третий).
Тогда исходная оболочка у тебя содержала векторы высоты 5, но в тот момент, когда ты решила делать преобразования
\textbf{не того же типа}, каким записала векторы в матрицу, ты изменила линейную оболочку, и начала решать какую-то другую
задачу.\\
Резюме: если ты предпочитаешь всегда делать преобразования строк, то в задаче, в которой надо \textbf{найти базис данной линеной 
оболочки}, тогда следует записывать векторы в строки матрицы.\\
2) \textbf{Среди данных} векторов нужно {выбрать} базис подпространства, натянутого на линейную оболочку. В этом
случае следует записывать векторы в матрицу одним типом, а преобразования делать другого типа. Например, если ты снова хочешь
делать только преобразования строк, то следует записать векторы в столбцы. Тогда, приведя полученную матрицу к ступенчатому
виду, ты получишь \textbf{номера векторов}, которые будут базисов в \textbf{исходной линейной оболочке}. Как получить эти номера?
Ща покажу:\\
Пусть у нас даны векторы, которые уже записаны в столбцы матрицы. Тогда выполняя элементарные преобразования строк
($\rightsquigarrow$) мы приводим матрицу к ступенчатому виду:
\[
\begin{pmatrix}
1 & 0 & 0 & 1\\
0 & 1 & 0 & -1\\
1 & 0 & 1 & 2\\
-1 & 2 & 3 & 0\\
\end{pmatrix}
\rightsquigarrow
\begin{pmatrix}
1 & 0 & 0 & 1\\
0 & 1 & 0 & -1\\
0 & 0 & 1 & 1\\
0 & 0 & 0 & 0\\
\end{pmatrix}
\]
На ступеньках расположились векторы $(1\ 0\ 0\ 0), (0\ 1\ 0\ 0), (0\ 0\ 1\ 0)$, но искомыми базисными будут не они,
а соответсвенно первый, второй и третий столбец в исходной матрице, т.е. векторы $(1\ 0\ 1\ -1), (0\ 1\ 0\ 2), (0\ 0\ 1\ 3)$.\\
Если честно, я понятия не имею, почему ломая линейную оболочку мы получаем исходный базис, хочу задать этот вопрос
Авдееву...\\
3) Даны векторы, надо проверить их на линейную независимость. Ну, это пожалуй самая простая задача и проверяется по определению.
Рассмотрим на конкретном примере: пусть даны векторы
\[
b_1 = 
\begin{pmatrix}
1\\
-1\\
0\\
\end{pmatrix},
b_2 = 
\begin{pmatrix}
2\\
-2\\
1\\
\end{pmatrix},
b_3 = 
\begin{pmatrix}
1\\
-1\\
1\\
\end{pmatrix}
\]
Когда набор векторов линейно зависим? По определению: когда существует нетривиальная линейная комбинация этих векторов, 
т.е. когда существует ненулевой набор скаляров $(a_1 a_2 a_3)$, такой что $a_1b_1 + a_2b_2 + a_3b_3 = 0$. Будем искать такой набор
скаляров:
\[
a_1
\begin{pmatrix}
1\\
-1\\
0\\
\end{pmatrix}
+ a_2
\begin{pmatrix}
2\\
-2\\
1\\
\end{pmatrix}
+ a_3
\begin{pmatrix}
1\\
-1\\
1\\
\end{pmatrix}
= 0
\Leftrightarrow
\begin{pmatrix}
1 & 2 & 1\\
-1 & -2 & -1\\
0 & 1 & 1\\
\end{pmatrix}
\begin{pmatrix}
a_1\\
a_2\\
a_3\\
\end{pmatrix}
=0
\]
Получили СЛУ, записываем расширенную матрицу и приводим её к улучшенному ступенчатому виду:
\[
\begin{pmatrix}
\begin{tabular}{c c c | c}
1 & 2 & 1 & 0\\
-1 & -2 & -1 & 0\\
0 & 1 & 1 & 0\\
\end{tabular}
\end{pmatrix}
\rightsquigarrow
\begin{pmatrix}
\begin{tabular}{c c c | c}
1 & 0 & -1 & 0\\
0 & 1 & 1 & 0\\
\end{tabular}
\end{pmatrix}
\]
Записываем общее решение: $
\left\{
\begin{pmatrix}
a_3\\
-a_3\\
a_3\\
\end{pmatrix}
, a_3 \in \mathbb{R}
\right\}$. Получилось, что существует не только нулевое решение, а значит исходный набор векторов линейно зависим.
\\
\\
\section{№ 5.1}
Доказать, что векторы $v_1 = (1\ -2\ -1\ 2\ 1), v_2 = (3\ 6\ 1\ 1\ 5), v_3 = (2\ -1\ -1\ a\ 1) \in \mathbb{R}^5$ линейно независимы
при всех значениях параметра $a$ и для каждого значения $a$ дополнить вектры до базиса в $\mathbb{R}^5$.
\section{№ 5.2}
Доказать, что векторы $v_1 = (1\ -1\ -1\ 2\ 7), v_2 = (3\ 1\ 5\ 7\ -1), v_3 = (2\ -3\ -4\ a\ 2) \in \mathbb{R}^5$ линейно независимы
при всех значениях параметра $a$ и для каждого значения $a$ дополнить вектры до базиса в $\mathbb{R}^5$.
\section{№ 5.3}
В пространстве $\mathbb{R}^3$ даны векторы
\[
v_1 = (5\ 3\ 7), v_2 = (-1\ 1\ -2), v_3 = (3\ 5\ 3), v_4 = (3\ 4\ 2), v_5 = (3\ 7\ 5)
\]
Выбрать среди данных векторов базис их линейной оболочки и для каждого вектора, не вошедшего в базис найти его
линейное выражение через базисные векторы.
\section{№ 5.4}
В пространстве $\mathbb{R}^3$ даны векторы
\[
v_1 = (5\ 3\ 7), v_2 = (2\ -1\ 2), v_3 = (-1\ 6\ 1), v_4 = (3\ -3\ 1), v_5 = (7\ 3\ 5)
\]
Выбрать среди данных векторов базис их линейной оболочки и для каждого вектора, не вошедшего в базис найти его
линейное выражение через базисные векторы.
\\
\\
\textbf{Ещё немного про задачи векторного пространства.}\\
1) В задаче может требоваться найти ранг матрицы, что это за зверь, и с чем его едят?\\
Говоря понятным языком, ранг матрицы, это число ненулевых строк в ступенчатом виде матрицы, и он же - число главных переменных
в матрицы коэффициентов ОСЛУ. На лекции была теоремка о том, что любые элементарные преобразования не меняют ранг матрицы,
и это на самом деле очень классная теорема, потому что в задаче, где нам надо искать ранг системы векторов, мы абсолютно пофиг,
чё с этой системой делать. Можем записать в строки матрицы, можем в столбцы. Можем делать преобразования строк, можем делать
преобразования столбцов, можем даже их комбинировать, и всё равно ранг от этого не поменяется.\\
\\
2) Ну, чтобы не вставать дважды, обсудим сразу ФСР, потому как оно и само по себе штука полезная, и рангом связана конкретным образом.
Прежде всего ФСР не существует просто так, он следует вместе с однородной СЛУ, и рассматривать их отдельно бессмысленно.\\
Пусть у нас есть какая-то СЛУ $Ax = 0$. Мы знаем, что множество её решений задаёт подпространство в $F^n$. При этом
сразу заметим и \textbf{запомним} важный факт: размерность 
\end{document}
