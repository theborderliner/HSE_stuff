\documentclass[a4paper,11pt]{report}

\usepackage[T2A]{fontenc}
\usepackage[utf8]{inputenc}
\usepackage[english,russian]{babel}
\usepackage{amsmath} %математические пакеты
\usepackage{amsfonts}
\usepackage{amssymb}
\usepackage[pdftex,unicode]{hyperref}
\usepackage[top=2cm,
left=2cm,
right=2cm,
bottom=2cm]{geometry}

\title{Макроэкономика}
\author{Пешехонов Иван. БПМИ1912}
\date{\today}

\begin{document}
\section{Модель совокупного спроса и предложения}
Совокупный спрос - сумма всех стпросов, который предъявляют все макроэкономические агенты (обозначается $AD$).\\
$AD = C + I + G + Xn$\\
Величиная совоеупного спроса - объём конечных товаров и услуг, которые готовы приоберсти макроэкономические агенты.\\


\end{document}          
