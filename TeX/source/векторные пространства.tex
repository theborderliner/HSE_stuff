\documentclass[a4paper,11pt]{report}

\usepackage[T2A]{fontenc}
\usepackage[utf8]{inputenc}
\usepackage[english,russian]{babel}
\usepackage{amsmath} %математические пакеты
\usepackage{amsfonts}
\usepackage{amssymb}
\usepackage[pdftex,unicode]{hyperref}
\usepackage{ulem}
\usepackage[top=2cm,
left=2cm,
right=2cm,
bottom=2cm]{geometry}

\title{Линал. Семинар. Векторные пространства.}
\author{Пешехонов Иван. БПМИ1912}
\date{\today}

\DeclareMathOperator{\Mn}{{\rm I\!R}^{n}}

\begin{document}
\chapter{Теория векторных пространств}
\section{Свойства векторных пространств}
\textbf{Свойство:} $(-1) * x = -x$\\
\textbf{Доказательсво:}\\
$x + (-1) * x = 0\\
1 * x + (-1) * x = 0\\
(1 + (-1)) * x = 0\\
1 + (-1) = 0$ (по свойству поля) $\Rightarrow 0 * x = 0$
\\
V = $
\begin{cases}
\begin{tabular}{1|1}
x \in \mathbb{R}^4 & 
\begin{gathered}
x_1 - x_2 + x_3 = 0\\\hfill
x_2 - x_4 = 0\\
\end{gathered}
\end{tabular}
\end{cases}\\
$\\
Базис - система векторов, через которую можно однозначно выразить любой вектор.\\
\section{Линейная зависимость\textbackslashнезависимость}
Набор векторов $(v_1, \cdots, v_n)$ называется линейно независимым, если $a_1v_1 + \cdots + a_nv_n = 0 \Rightarrow a_1 = \cdots = a_n = 0$.\\
И набор векторов $(v_1, \cdots, v_n)$ называется линейно зависимым, если $\exists (a_1, \cdots, a_n) \neq (0, \cdots, 0)$\\
\\
\section{Ранг системы векторов.}
\section{Базис}
$(v_1, \cdots, v_n)$ - базис пространства $V$ $\Leftrightarrow$\\
$\Leftrightarrow \forall w \in V $ $\exists!$ $ a_1, \cdots, a_n \Rightarrow w = a_1v_1 + \cdots + a_nv_n$\\
2) Максимально линейно независимая система.\\
3) Минимальная подсистема, через которую любой вектор линейно выражается.\\
\\
\textbf{Размерностью линейного пространства} называется количество векторов в его базисе.\\
1) $\Mn$\\
2) Cимметричные матрицы $\Mn$\\
$A = \sum_i{a_{ii}E_{ii}} + \sum_{i < j}{a_{ij}(E_{ij} + E_{ji})}$\\
\\
3)
$V = \langle 
\begin{tabular}{cccc}
1 & 0 & 0 & 1\\
0 & 1 & 0 & -1\\
1 & 0 & 1 & 2\\
-1 & 2 & 3 & 0\\
\end{tabular}
\rangle
=
\begin{pmatrix}
1 & 0 & 0 & 1\\
0 & 1 & 0 & -1\\
1 & 0 & 1 & 2\\
-1 & 2 & 3 & 0\\
\end{pmatrix}
= (*)
$\\
Дальше элементарными преобразованиями строк приводим $(*)$ к ступенчатому виду\\
\[
\begin{pmatrix}
1 & 0 & 0 & 1\\
0 & 1 & 0 & -1\\
0 & 0 & 1 & 1\\
0 & 0 & 3 & 3\\
\end{pmatrix}
=
\begin{pmatrix}
1 & 0 & 0 & 1\\
0 & 1 & 0 & -1\\
0 & 0 & 1 & 1\\
\end{pmatrix}
\]\\
Базисом в такой матриц будут векторы-ступеньки.\\
\\
$\lambda_1v_1 + \lambda_2v_2 + \lambda_3v_3 = 0 \Leftrightarrow$
$ \lambda_1(Uv_1) + \lambda_2(Uv_2) + \lambda_3(Uv_3) = 0$, где $U$ - матрица преобразований.\\
\textbf{Вывод:} элементарные преобразования не меняют линейную зависимость.\\
\\
Рассмотрим другой путь:\\
\[
(*) =
\begin{pmatrix}
1 & 0 & 0 & 1\\
0 & 1 & 0 & -1\\
1 & 0 & 1 & 2\\
-1 & 2 & 3 & 0\\
\end{pmatrix}
\]\\
Выполняем элементарные преобразования столбцов\\
Когда выполняем преобразования столбцов, мы не выходим из линейной оболочки, и переходим 
к линейной комбинации столбцов.\\
\underline{Элементарные преобразования слолбцов дают базис!}\\
Элементарные преобразования строк возвращают номера.... какие?
\section{Консультация}
\subsection{Дополнение и разъяснение к семинару.}
Есть 4 варианта, как нужно действовать при работе с элементами викторного пространства (векторами), когда хочется найти базис векторного пространства:
1) Записать векторы в столбцы и делать элементарные преобразования столбцов.\\
2) Записать векторы в столбцы и делать элементарные преобразования строк.\\
3) Записать векторы в строки и делать элементарные преобразования столбцов.\\
4) Записать векторы в строки и далать элементарные преобразования строк.\\

Если записать векторы в столбцы и делать элементарные преобразования строк, то будут сохраняться линейные зависимости между векторами.\\
\\
Дано: $V$ - векторное пространство, $S$ - его линейная оболочка.
Базис $V$ это такой набор векторов, для которого выполняются два условия:
1) Все вектора в базисе линейно-независимы.
2) Базисные вектора порождают $S$. (Вероятно, имеется ввиду, что $S$ - линейная оболочка, по определеню множество всех 
линейных комбинаций векторов в $V$, то т.к. каждый вектор из $V$ можно представить в виде линейной комбинации
базисных векторов, то и каждую линейную комбинацию из $S$ (каждый элемент множества $S$) можно представить в виде 
линейной комбинации базисных векторов. Видимо, именно поэтому базисные вектора порождают $S$, ну или я хз, чё он ещё мог иметь ввиду.)\\
\\
\textbf{Утверждение:} \\
Дана матрица $A =  (A^{(1)} A^{(2)} \cdots  A^{(n)})$\\
Пусть, в результате применения \underline{одного} элементарного преобразования строк получилась матрица $B = (B^{(1)} B^{(2)} \cdots  B^{(n)})$.\\
Утверждается, что $a_1A^{(1)} + ... + a_nA^{(n)} = 0 \Leftrightarrow a_1B^{(1)} + ... + a_nB^{(n)} = 0$\\
\textbf{Доказательсво:} \\
$a_1A^{(1)} + ... + a_nA^{(n)} = 0 \Leftrightarrow A
\begin{pmatrix}
a_1\\
\vdots\\
a_n
\end{pmatrix}
= 0
\Leftrightarrow 
\begin{pmatrix}
a_1\\
\vdots\\
a_n
\end{pmatrix}
$ - решения ОСЛУ $Ax = 0 \Leftrightarrow
\begin{pmatrix}
a_1\\
\vdots\\
a_n
\end{pmatrix}
$ - решения ОСЛУ $Bx = 0 \Leftrightarrow B
\begin{pmatrix}
a_1\\
\vdots\\
a_n
\end{pmatrix}
= 0 \Leftrightarrow a_1B^{(1)} + ... + a_nB^{(n)} = 0
$ $ \blacksquare$\\
\\
Есть два конкретных алгоритма, как решать задачи на поиск базиса, выбор того или иного алгоритма зависит от формулировки задачи:\\
1) В задаче нужно \textbf{выбрать} базис линейной оболочки из данных векторов. Алгоритм:\\
\begin{itemize}
     \item Записываем данные векторы в столбцы матрицы.
     \item С помощью элементарных преобразований \textbf{строк} приводим матрицу к ступенчатому виду.
     \item Смотрим на векторы, расположившиеся на ступеньках. Порядковые номера этих векторов будут указывать на
     базисные вектора в \underline{исходной матрице}
\end{itemize}\\
Пример: (через $\rightsquigarrow$ я обозначу элементарные преобразования строк)\\
Пусть у нас есть некоторые векторы, которые уже записаны в столбцы матрицы, к которой мы применяем элементарные преобразования:\\
\[
\begin{pmatrix}
1 & 0 & 0 & 1\\
0 & 1 & 0 & -1\\
1 & 0 & 1 & 2\\
-1 & 2 & 3 & 0\\
\end{pmatrix}
\rightsquigarrow
\begin{pmatrix}
1 & 0 & 0 & 1\\
0 & 1 & 0 & -1\\
0 & 0 & 1 & 1\\
0 & 0 & 3 & 3\\
\end{pmatrix}
\]
На ступеньках расположились векторы $(1\ 0\ 0\ 0), (0\ 1\ 0\ 0), (0\ 0\ 1\ 3)$, но искомыми базисными будут не они, а те векторы,
которые как бы стоят на их местах в исходной матрице, т.е. $(1\ 0\ 1\ {-1}), (0\ 1\ 0\ 2), (0\ 0\ 1\ 3)$. 
И все три этих вектора образуют искомый базис, а не каждый из них в отдельности.\\
\\
2) В задаче нужно \textbf{найти какой-нибудь} базис. Алгоритм:\\
\begin{itemize}
     \item Записываем данные векторы в строки матрицы.
     \item С помощью элементарных преобразований \textbf{строк} приводим матрицу к ступенчатому виду.
     \item Смотрим на векторы, расположившиеся на ступеньках.\\
     \underline{Эти векторы образуют базис линейной оболочки}.
\end{itemize}\\

\subsection{Как доказать линейную независимость?}
Общий алгоритм:\\
$a_1v_1 + ... +a_kv_k = 0 \Leftrightarrow a_1 = ... = a_k = 0$ (Это надо просто взять и \sout{постигнуть} увидеть. 
Серьёзно, часто придётся действовать именно так, без каких-то конкретных указаний).\\
Но, если мы имеем векторы, такие что $v_i \in \mathbb{R}^n$,
то можно составить ОСЛУ из $v_i и a_i$ и решить её.
\chapter{Семинар Авдеева}
$Ax = b (*), A \in \mathbb{F}^n$\\
$\det{A} \neq 0 \Rightarrow (*) \Leftrightarrow x = A^{-1}b$\\
Линейная оболочка множества $\Leftrightarrow$ пространство, порождённое множеством (пространство, натянутое множеством).
$V$ - векторное пространство над $F$\\
$v_1, ..., v)n \in V$\\
Система векторов линейно зависима, если $\exists (a_1 ... a_n) \neq (0, ..., 0)$ такая что $a_1v_1 + ... + a_nv_n = 0$\\
И система векторов линейно независима, если $a_1v_1 + ... + a_nv_n = 0 \Leftrightarrow a_1 =  ... = a_n = 0$\\
$v_1 ... v_n \in V = F^n$\\
$a_1v_1 + ... + a_nv_n  = 0 \Leftrightarrow A*(a_1 ... a_n) = 0, a = (v_1 ... v_n)$\\
Алгоритм:\\
1. Запишем столбцы в матрицу\\
2. Метод Гауса (до ступенчатого вида)\\
\textbf{ОСЛУ имеет единственное решение $\Leftrightarrow$ она имеет больше одной свободной неизвестной.}\\
\\
$V=\{$ - все ф-ции $R -> R\}$\\
Возьмём в качестве векторов $\sin{x}, \sin^2{x}, ..., \sin^n{x}$, вопрос, линейно зависима ли эта система?\\
Линейно зависима $\Leftrightarrow \exists (a_1 ... a_n) \neq (0, ..., 0)$ такая что $a_1\sin{x} + a_2\sin^2{x} + ... + a_n\sin^n{x} = 0$, 
где $0$ - ф-ция, которая равна $0$ при любом входе.\\
\end{document}
